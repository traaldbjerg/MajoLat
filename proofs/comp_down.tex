\begin{lemma} \label{lem:comp_down} (\cite{cicalese_information_2013})
    The join $p \vee q$ of two probability vectors $p$ and $q$ saturates the triangular inequality for the distance defined as $d(p, q) \coloneqq H(p) + H(q) - 2H(p \vee q)$.
\end{lemma}

\begin{proof}

\begin{align*}
    d(p, p \vee q) + d(p \vee q, q) &= H(p) + H(p \vee q) - 2H(p \vee (p \vee q)) + H(p \vee q) + H(q) - 2H((p \vee q) \vee q) \\
    &= H(p) + H(p \vee q) - 2H(p \vee q) + H(p \vee q) + H(q) - 2H(p \vee q) \\
    &= H(p) + H(q) - 2H(p \vee q) \\
    &= d(p, q).
\end{align*} \phantom{\qedhere}

\end{proof}

\begin{result}
    The measure of comparability of a vector $p$ to the majorizing cone of a vector $q$ defined as $E_q^- (p) \coloneqq \displaystyle{\min_{s \succ q} d(p, s)}$ is equal to the distance $d(p, p \vee q)$.
\end{result}

\begin{proof}

\begin{align*}
    E_q^- (p) &= \min_{s \succ q} d(p, s) \\
    \overset{\text{Lemma \ref{lem:comp_down}}}&{=} \min_{s \succ q} d(p, p \vee s) + d(p \vee s, s)\\
\end{align*}

\noindent Let $s' = p \vee s$. The absorption law $a \vee a = a$, along with the associativity of the join operation, implies that choosing $s = s'$ (i.e. requiring that $s$ majorizes $p$) makes the second term of the sum vanish, while the first remains invariant.

\begin{align*}
    \implies E_q^- (p) &= \min_{s \succ q} d(p, p \vee s') + d(p \vee s', s') \\
    &= \min_{s \succ q} d(p, p \vee (p \vee s)) + d(p \vee (p \vee s), p \vee s) \\
    &= \min_{s \succ q} d(p, (p \vee p) \vee s) + d((p \vee p) \vee s, p \vee s)\\
    &= \min_{s \succ q} d(p, p \vee s) + d(p \vee s, p \vee s)\\
    &= \min_{s \succ q} d(p, p \vee s)
\end{align*}

\noindent Because $s$ majorizes $q$, $s'$ majorizes $q$ too. Moreover, to be the join of $p$ with another vector, $s'$ must majorize $p$ as well.

\begin{align*}
    \implies E_q^- (p) &= \min_{s' \succ q, p} d(p, s') \\
    &= \min_{s' \succ q, p} H(p) + H(s') - 2H(p \vee s') \\
    &= \min_{s' \succ q, p} H(p) + H(s') - 2H(s') \\
    &= \min_{s' \succ q, p} H(p) - H(s').\\
\end{align*}

\noindent The Schur-concavity of the Shannon entropy implies that the maximum value of $H(s')$ in the subset of $P^n$ $\{v \mid v \succ p, q\}$ is reached for a vector $s'$ that is majorized by all other vectors in the subset. By definition, that vector is the join $p \vee q$. \qedhere

\end{proof}