\chapter{Conclusion}

In this MSc thesis, we have mainly defined the foundations of a mathematical theory of incomparability by taking inspiration from resource-theoretic approaches.

In Chapter \ref{chap:alternative}, we have proven some more general forms of the properties of supermodularity and subadditivity -- which Shannon entropy was known to enjoy on the majorization lattice -- to all sum-concave functions. The concatenation technique used to prove the underlying majorization precursor seems fairly powerful, and might yield interesting majorization precursors in other fields that deal with entropic inequalities.

In Chapter \ref{chap:criteria}, we have gone over a small improvement to coupled entropic criteria for which a majorization relation already exists, notably the majorization separability criterion. This criterion is only an improvement for incomparable Schmidt vectors, and so we defined new quantities to measure the incomparability of two distributions, and studied some of their properties under entanglement transformations.

Finally, in Chapter \ref{chap:volume}, the ideas from Chapter \ref{chap:incomparability} came to fruitition as we succesfully generalized the future entropy monotone into the uniqueness entropy, yielding new geometrical intuitions explaining some of the properties of the Shannon entropy on the majorization lattice. We didn't prove a rigorous measure, but we were recently made aware of the relevance of Heyting lattices, on which many measures have already been defined, from which inspiration might be taken to define our own measure. We have also found an application to the uniqueness entropy, and shown that some more general forms of resource monotones which take into account all of the states in our possession and not only each individual state separately yields better results in some situations, giving the idea that on some level it really is a valuable notion of resource in the quantum picture. As such, we have shown that there is some merit to such approaches, and that the lattice is a useful tool to quantify collective properties. Moreover, while the improvements over a simple entropic strategy are not very large, one has to keep in mind that some of the choices made to define the different RSSS were fairly arbitrary, so we believe that further study of lattice-based quantities and state filters might yield better results. We were recently made aware of the relevance of the field of dynamic optimization theory, which is precisely concerned with such resource-selection decision algorithms.

We believe that further research in defining other RSSS, either strictly algorithmically or by defining other lattice-based quantities might yield better results. The example with the successive target constructions might have been a little artificial, but it serves as a proof-of-concept for this new approach, and we believe that reversing the discussion to define a RSSS for OCR construction might be a more practically relevant case (although more complicated because of probabilistic transformations). In another direction, we also believe that further research on an entropic measure could be interesting, as such a result could lead to the discovery of new properties of the Shannon entropy. Finally, the foundations of the mathematical theory of incomparability that we have defined in this manuscript could be applied to other QRTs than entanglement theory, which might be another fruitful research avenue.