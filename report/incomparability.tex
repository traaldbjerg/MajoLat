\chapter{Quantification of incomparability} \label{chap:incomparability}

This chapter tries to quantify the amount of incomparability between probability vectors. The idea of proceeding this way came from chapter \ref{chap:criteria}, where incomparability seemed to allow us to improve entropic criteria. As such, studying how incomparable two distributions are seemed an interesting prospect.

\section{Entropic distance approach}

\subsection{Resource-theoretic intuition}

The approach that was taken was to take inspiration from resource theories to define incomparability monotones. What is often done in resource theories is to define a monotone as being the minimum distance from the studied state to the set of free states. For instance, the entanglement monotone $H(\lambda_\psi)$ is such a monotone. Consider the entropic distance (cf. section \ref{sec:entropic_distances}) $d(\lambda_\psi, \lambda_{\text{free}})$. We know that all minimally entangled states have $\lambda_\text{free} = (1, 0, \dots, 0)$ (they are separable), and so the entropic distance reduces to

\begin{align}
    d(\lambda_\psi, \lambda_\text{free}) &= H(\lambda_\psi) + H(\lambda_\text{free}) - 2 H(\lambda_\psi \vee \lambda_\text{free})\\
                                      &= H(\lambda_\psi),
\end{align}

\noindent where $H(\lambda_\text{free}) = 0$ (the certain distribution has no entropy) and $\lambda_\psi \vee \lambda_\text{free} = \lambda_\text{free}$ because the certain distribution is the bottom of the lattice. In this particular case, there is no need to take the minimum over all free states because they all have the same Schmidt vector, but in other resource theories or using other notions of distance between quantum states, they might not all be at the same distance from the state $\lambda_\psi$.

However, there are some key differences when quantiyfing incomparability. First, the set of comparable vectors to a single vector $q$ is not convex. Figure \ref{fig:meet_join_incomp} illustrates why this is the case: a convex mixture of a vector from $\mathcal{T}_+(q)$ and $\mathcal{T}_-(q)$ could land in $\mathcal{T}_\emptyset(q)$. % ?? check le souci enft je vois pas exactement ce qui va pas
To avoid this problem, we will propose two different notions of incomparability.

\begin{figure}[h!] % rajouter les 4 distances sur la figure ?
    \centering
    \begin{tikzpicture}[scale=0.9]
        % draw cone of p
        \coordinate (A) at (-2,3);
        \coordinate (B) at (2, -3);
        \draw [name path=A--B] (A) -- (B);
        \coordinate (C) at (-2,-3);
        \coordinate (D) at (2,3);
        \draw [name path=C--D] (C) -- (D);
        \path [name intersections={of=A--B and C--D,by=E}];
        \node [fill=black,inner sep=2pt,label=0:$p$] at (E) {};
        % draw cone of q
        \coordinate (F) at (-0.666,3);
        \coordinate (G) at (3.333, -3);
        \draw [name path=F--G] (F) -- (G);
        \coordinate (H) at (0.666,-3);
        \coordinate (I) at (4.666,3);
        \draw [name path=H--I] (H) -- (I);
        \path [name intersections={of=F--G and H--I,by=J}];
        \node [fill=black,inner sep=2pt,label=0:$q$] at (J) {};
        %intersections of the 2 cones
        \path [name intersections={of=F--G and C--D,by=K}];
        \path [name intersections={of=H--I and A--B,by=L}];
        \node [fill=black,inner sep=2pt,label=0:$p \wedge q$] at (K) {};
        \node [fill=black,inner sep=2pt,label=0:$p \vee q$] at (L) {};
        %fill the two cones
        %\fill[fill=blue, opacity=0.2] (F) -- (K) -- (D) -- cycle;
        %\node [inner sep=0pt] at ($ (F) !.3333! (K) !.3333! (D) $) {$\mathcal{T}_- (p \wedge q)$};
        %\fill[fill=red, opacity=0.2] (H) -- (L) -- (B) -- cycle;
        %\node [inner sep=5pt, label=-90:$\mathcal{T}_+ (p \vee q)$] at ($ (H) !.3333! (L) !.3333! (B) $) {};
        
    \end{tikzpicture}
    \caption{Depiction of incomparable vectors $p, q \in \mathcal{P}^d$, and of their meet $p \wedge q$ and join $p \vee q$.}
    \label{fig:meet_join_incomp}
\end{figure}

The second major difference is that a vector can only be incomparable to another vector (a vector is not incomparable in and of itself). As such, treating incomparability as a resource is a bit strange because it would characterize some form of collective property of a pair of states, rather than an intrinsic property of states themselves which most resource theories do. In the following, we will call $p$ the \textit{probe state}, and $q$ the \textit{reference state} that $p$ is compared to. Of course, $q$ can also be incomparable to $p$, and so nothing prevents us from reversing the roles of $p$ and $q$ to study the other half of the collective property.



\subsection{Future incomparability function}

Considering the previous discussion, there are 2 notions of incomparability we can define. The first will be called the notion of \textit{future incomparability}. Essentially, it is simply going to characterize how far $p$ is from the future cone of $q$. We propose the following definition.

\begin{definition}[Future incomparability function] \label{def:future_incomparability_monotone}
    Let $p, q \in \mathcal{P}^d$. The future incomparability $E^+(p \parallel q)$ of $p$ to $q$ is defined as
    \begin{equation}
        E^+ (p \parallel q) = \min_{s \succ q} d(p, s).
    \end{equation}
\end{definition}

Figure FAIRE FIGURE illustrates this notion of incomparability directly on the lattice. Essentially, we are looking for the vector $s$ in $\mathcal{T}_+(q)$ which is closest to $p$ in terms of entropic distance. Intuitively, we would expect this vector to be $p \vee q$. To prove this is the case, we will first need the following lemma.

\begin{lemma} (Cicalese, Gargano and Vaccaro, 2013 \cite{cicalese_information_2013}) \label{lem:comp_future} 
    The join $p \vee q$ of two probability vectors $p, q \in \mathcal{P}^d$ saturates the triangular inequality for the entropic distance $d(p, q) = H(p) + H(q) - 2H(p \vee q)$.
\end{lemma}

\begin{proof}
    \begin{align}
        d(p, p \vee q) + d(p \vee q, q) &= H(p) + H(p \vee q) - 2H(p \vee (p \vee q)) + H(p \vee q) \nonumber \\
        &\quad \quad + H(q) - 2H((p \vee q) \vee q) \\
        &= H(p) + H(p \vee q) - 2H(p \vee q) + H(p \vee q) + H(q) \nonumber\\
        &\quad \quad - 2H(p \vee q) \\
        &= H(p) + H(q) - 2H(p \vee q) \\
        &= d(p, q),
    \end{align}
    where we have used associativity of successive joins to simplify the expressions. \phantom{\qedhere}
\end{proof}

This lemma gives us the tools to prove the first major result of this chapter.

\begin{theorem}
    The future incomparability of a vector $p$ to another vector $q$ is the entropic distance from $p$ to their join. Formally,
    \begin{equation}
        E^+ (p \parallel q) = d(p, p \vee q).
    \end{equation}
\end{theorem}

Essentially, this theorem simply states that our geometric intution concerning majorization cones is correct (in this case). Let us now prove this result.

\begin{proof}
    Let us find the minimal value of $d(p, s)$ (with $s \succ q$), and let us show that it is realized for $s = p \vee q$.

    \begin{align}
        E^+(p \parallel q) &= \min_{s \succ q} d(p, s) \\
        \overset{\text{Lemma \ref{lem:comp_future}}}&{=} \min_{s \succ q} d(p, p \vee s) + d(p \vee s, s)\\
    \end{align}

    \noindent Let $s' = p \vee s$. The absorption law $a \vee a = a$, along with the associativity of the join operation, implies that choosing $s = s'$ (i.e. requiring that $s$ majorizes $p$) makes the second term of the sum vanish, while the first remains invariant.

    \begin{align}
        \implies E^+ (p \parallel q) &= \min_{s \succ q} d(p, p \vee s') + d(p \vee s', s') \\
        &= \min_{s \succ q} d(p, p \vee (p \vee s)) + d(p \vee (p \vee s), p \vee s) \\
        &= \min_{s \succ q} d(p, (p \vee p) \vee s) + d((p \vee p) \vee s, p \vee s)\\
        &= \min_{s \succ q} d(p, p \vee s) + d(p \vee s, p \vee s)\\
        &= \min_{s \succ q} d(p, p \vee s)
    \end{align}

    \noindent Because $s$ majorizes $q$, $s'$ majorizes $q$ too. Moreover, to be the join of $p$ with another vector, $s'$ must majorize $p$ as well.

    \begin{align}
        \implies E^+ (p \parallel q) &= \min_{s' \succ q, p} d(p, s') \\
        &= \min_{s' \succ q, p} H(p) + H(s') - 2H(p \vee s') \\
        &= \min_{s' \succ q, p} H(p) + H(s') - 2H(s') \\
        &= \min_{s' \succ q, p} H(p) - H(s').\\
    \end{align}

    \noindent The Schur-concavity of the Shannon entropy implies that the maximum value of $H(s')$ for $s' \in \{v \mid v \succ p, q\}$\footnote{This is equivalent to saying $s' \in \mathcal{T}_+(p) \cap \mathcal{T}_+(q)$.}$ \subseteq \mathcal{P}^d$ is reached for a vector $s'$ that is majorized by all other vectors in the subset. By definition, that vector is the join $p \vee q$. \qedhere
\end{proof}



\subsection{Past incomparability function}

One could attempt the same definition for a \textit{past incomparability} function of $p$ to $q$ as being the minimal distance from $p$ to the past cone of $q$, $\mathcal{T}_-(q)$. However, using the same entropic distance does not yield properties as nice as we had with the join. A simple interpretation of this would be the hyperbolic geometry induced by our entropic distance, which complicates any attempts at working with the meet. Instead, we tried working with the entropic quasidistance $d'(p, q) = 2H(p \wedge q) - H(p) - H(q)$ as introduced in definition \ref{def:entropic_quasidistance}. We propose the following definition.

\begin{definition}[Past incomparability function]
    Let $p, q \in \mathcal{P}^d$. The past incomparability $E^-(p \parallel q)$ of $p$ to $q$ is defined as
    \begin{equation}
        E^- (p \parallel q) = \min_{s \prec q} d'(p, s).
    \end{equation}
\end{definition}

Figure FAIRE FIGURE illustrates this definition on the lattice. The same interpretation can be given, though one should be careful that we are working with a quasidistance this time around, and so geometric intuitions can be deceiving.

With this quasidistance, we fall on similar properties as we did with the future incomparability monotone. Namely, the minimal quasidistance to the past cone is reached for the meet $p \wedge q$. Before proving this, we will need the following lemma.

\begin{lemma} \label{lem:comp_past}
    $d'(p, q) = d'(p, p \wedge q) + d'(p \wedge q, q)$ for the entropic quasidistance $d'(p, q) = 2H(p \wedge q) - H(p) - H(q)$.
\end{lemma}

\begin{proof}
    \begin{align}
        d'(p, p \wedge q) + d'(p \wedge q, q) &= 2H(p \wedge (p \wedge q)) - H(p) - H(p \wedge q) + 2H((p \wedge q) \wedge q) \nonumber\\
        &\quad \quad - H(p \wedge q) - H(q)\\
        &= 2H(p \wedge q) - H(p \wedge q) - H(p) + 2H(p \wedge q) - H(p \wedge q)\nonumber\\
        &\quad \quad - H(q) \\
        &= 2H(p \wedge q) - H(p) - H(q) \\
        &= d'(p, q).
    \end{align} \phantom{\qedhere}
\end{proof}

We are now ready to prove the main result of this section, which again states that our geometric intuition is correct (though this time the interpretation is murkier given the quasidistance).

\begin{theorem}
    The past incomparability of a vector $p$ to another vector $q$ is the entropic quasidistance from $p$ to their meet. Formally,
    \begin{equation}
        E^- (p \parallel q) = d'(p, p \wedge q).
    \end{equation}
\end{theorem}

\begin{proof}
    Let us find the minimal value of $d'(p, s)$ (with $s \prec q$), and let us show it is realized for $s = p \wedge q$.
    \begin{align}
        E^- (p \parallel q) &= \min_{s \prec q} d'(p, s) \\
        \overset{\text{Lemma \ref{lem:comp_past}}}&{=} \min_{s \prec q} d'(p, p \wedge s) + d'(p \wedge s, s) \\ 
        &\implies \min_{s \prec q} d'(p, s) \geq \min_{s \prec q} d'(p, p \wedge s) \label{eq:1}
    \end{align}

    \noindent Let $s' = p \wedge s$. Because $s$ is majorized by $q$, $s'$ is majorized by $q$ as well. Moreover, to be the meet of $p$ with another vector, $s'$ must be majorized by $p$ as well.

    \begin{align}
        \implies \min_{s \prec q} d'(p, p \wedge s) &= \min_{s' \prec q, p} d'(p, s')\\
        &= \min_{s' \prec q, p} 2H(p \wedge s') - H(p) - H(s')\\
        &= \min_{s' \prec q, p} H(s') - H(p).
    \end{align}

    \noindent The Schur-concavity of the Shannon entropy implies that the minimum value of $H(s')$ for  $s' \in \{v \mid v \prec p, q\}$\footnote{This is equivalent to saying $s' \in \mathcal{T}_-(p) \cap \mathcal{T}_-(q)$.}$ \subseteq \mathcal{P}^d$ is reached for a vector $s'$ that majorizes all other vectors in the subset. By definition, that vector is the meet $p \wedge q$, and so $\min_{s' \prec q, p} H(s') - H(p) = H(p \wedge q) - H(p)$. The vector $p \wedge q$ is also part of the original subset $\{v \mid v \prec q\}$, and so plugging $s = p \wedge q$ into (\ref{eq:1}) shows that the LHS realizes the value $H(p \wedge q) - H(p)$ as well, and the 2 minima must therefore be equal. \qedhere
\end{proof}



\section{Properties}

For this section, we will denote bistochastic matrices of dimension $d\times d$ by the letter $D$. Recall that an equivalent definition of majorization is $p \prec q \iff \exists D | q = Dp$ (cf. section \ref{sec:bistochastic}). Now that we have defined our two incomparability functions and shown that they are equal to the (quasi)distance to the meet or join, let us show that they are monotones under a bistochastic degradation. Conversely, if they are increasing (resp. decreasing) monotones under a bistochastic degradation, they are a decreasing (resp. increasing) monotone under a LOCC degradation in the quantum picture. However, we are working with 2 states, and so we can study degradations of $p$ and of $q$ separately.



\subsection{Monotonicity under bistochastic degradation of the probe} \label{sec:p_monotonicity}

Let us start with studying the future incomparability function. If we can show that $E^+(Dp \parallel q)$ is greater than $E^+(p \parallel q)$ for any bistochastic matrix $D$, then we can promote our future incomparability function $E^+$ to a monotone (under bistochastic degradation of $p$). Figure FAIRE FIGURE shows why we geometrically expect this to be the case. Moreover, this would also mean that $E^+$ can only decrease under LOCC of $p$. We will first need the following lemma.

\begin{lemma} \label{lem:incomparable_diamond}
    In any incomparable diamond $p, q, p \wedge q, p \vee q$ we have $d(p, p \vee q) \leq d(q, p \wedge q)$.
\end{lemma}

\begin{proof}
    \begin{align}
        d(p, p \vee q) &= H(p) + H(p \vee q) - 2H(p \vee (p \vee q))\\
        &= H(p) - H(p \vee q)\\
        &= H(p) + H(p \wedge q) - H(p \wedge q) - H(p \vee q)\\
        \overset{\text{supermod}}&{\leq} H(p \wedge q) - H(q)\\
        &= d(q, p \wedge q)
    \end{align} \phantom{\qedhere}
\end{proof}

This lemma is essentially just an equivalent way of stating supermodularity. We are now ready to prove the main theorem of this section.

\begin{theorem} \label{th:monotone_future_p}
    There exists no bistochastic matrix $D$ such that $E^+ (Dp \parallel q) < E^+ (p \parallel q)$.
\end{theorem}

This theorem essentially states that $E^+$ is an increasing monotone under bistochastic degradation of $p$. This proof is perhaps the trickiest of the chapter. Figure FAIRE FIGURE illustrates the different vectors of the construction on the lattice to help with comprehension.

\begin{proof}
    Let us show that the minimal value of $E^+ (p' \parallel q)$ (with $p' \prec p$) is realized for $p' = p$. Let $p'$ be a vector majorized by $p$, i.e. there exists a bistochastic matrix $D$ such that $p' = Dp$. Let $q'(p') = p' \vee q$, and let $p''(p') = p \wedge q'$ which are both functions of the variable to minimize over, $p'$. To avoid cluttering the expressions, we will simply write $p''$ and $q'$, but one should keep in mind that they are indeed functions of $p'$. By hypothesis and by definition of $q'$, $p'$ is majorized by both $p$ and $q'$, which is equivalent to $p'$ being majorized by $p \wedge q'$. We have

    \begin{equation}
         d(p'', q') = H(p \wedge q') - H(q') \leq H(p') - H(q') = d(p', q'), \label{eq:construction}
    \end{equation} 

    \noindent which can be used for the following development

    \begin{align}
        \min_{p' \prec p} E^+ (p' \parallel q) &= \min_{p' \prec p} d(p', p' \vee q)\\
        &= \min_{p' \prec p} d(p', q')\\
        \overset{\text{(\ref{eq:construction})}}&{=} \min_{p' \prec p} d(p'', q')\\
        &= \min_{p' \prec p} d(p \wedge q', q')\\
        \overset{\text{Lemma \ref{lem:incomparable_diamond}}}&{\geq} \min_{p' \prec p} d(p, p \vee q')\\
        &= \min_{p' \prec p} d(p, p \vee (p' \vee q))\\
        &= \min_{p' \prec p} d(p, (p \vee p') \vee q)\\
        &= d(p, p \vee q)\\
        &= E^+ (p \parallel q),
    \end{align}
    \noindent and so the minimal value of $E^+ (Dp \parallel q)$ is reached for the identity degradation which leaves $p$ invariant. \qedhere
\end{proof}
 
This theorem is quite satisfying in the sense that it seems to indicate that our geometric intuitions on the lattice are valid. Moreover, we can now promote the future incomparability function to a future incomparability \textit{monotone}. Theorem \ref{th:monotone_future_p} has a few corollaries.

\begin{corollary} \label{cor:incomparability_LOCC}
    Let $\ket{\psi}$ and $\ket{\phi}$ be two pure quantum states, and let $\lambda_\psi$ and $\lambda_\phi$ be the associated Schmidt vectors. If $\ket{\psi} \overset{\text{LOCC}}{\longrightarrow} \ket{\phi}$ with probability 1, then $E^+ (\lambda_\psi \parallel \lambda_\alpha) \geq E^+ (\lambda_\phi \parallel \lambda_\alpha)$ with $\lambda_\alpha$ some probability vector.
\end{corollary}

\begin{corollary} \label{cor:incomparability_separability}
    Let $\rho_{AB}$ be a bipartite quantum state, and let $\rho_A$ and $\rho_B$ be the reduced states. Let $\lambda_{AB}, \lambda_A$ and $\lambda_B$ be the vectors of eigenvalues of their density matrices. Then, if $\rho_{AB}$ is separable, $E^+ (\lambda_{AB} \parallel \lambda_B) \geq E^+ (\lambda_A \parallel \lambda_B)$ and $E^+ (\lambda_{AB} \parallel \lambda_A) \geq E^+ (\lambda_B \parallel \lambda_A)$.
\end{corollary}

These corollaries, while not very useful per se, still hold some interpretational value, in the sense that they show that in the sense of our future incomparability monotone, some states must be more incomparable than others. Corollary \ref{cor:incomparability_LOCC}, while involving some nondescript Schmidt vector $\lambda_\alpha$, essentially states that if an entangled state can reach another state through LOCC, then it also holds more incomparability to other states. Corollary \ref{cor:incomparability_separability} is perhaps more interesting, and states that if a joint state is separable, then the joint state is more incomparable to each of the reduced states than the reduced states are to each other.

One would hope that the analogue of theorem \ref{th:monotone_future_p} would hold for the measure of past incomparability $E^-$, however it does not. The following counterexample shows that a bistochastic matrix $D$ such that $E^- (Dp || q) < E^- (p || q)$ can exist, but also that a bistochastic matrix $D'$ such that $E^- (D'p || q) > E^- (p || q)$ can exist too. Let $q = (0.6, 0.4)$, $p = (0.7, 0.29, 0.1)$, $p' = p \wedge q$ and $p'' = (0.7, 0.15, 0.15)$. One can verify that $p$ majorizes both $p'$ and $p''$, yet $E^- (p || q) = 0.0939$ bits, $E^- (p' || q) = 0$ bits and $E^- (p'' || q) = 0.171$ bits.



\subsection{Monotonicity under bistochastic degradation of the reference} \label{sec:q_monotonicity}

Let us now turn our attention to bistochastic degradations of $q$, and attempt to promote our incomparability functions to monotones. Figure FAIRE FIGURE shows the monotonicity relations we expect. Starting with the future incomparability function, if we can show that $E^+(p \parallel Dq) < E^+(p \parallel q)$ for any bistochastic matrix $D$, then $E^+$ would be a decreasing monotone, like we expect. We first need a preliminar lemma.

\begin{lemma} \label{lem:monotone_future_q}
    For any $p, q \in \mathcal{P}^d$, we have
    \begin{equation}
        p \vee q \succ p \vee q' \; \; \; \forall \: q' \prec q.
    \end{equation}
\end{lemma}

\begin{proof}
    Let $q'$ be any probability vector majorized by $q \in \mathcal{P}^d$ and let $p \in \mathcal{P}^d$.
    \begin{align}
        p \vee q = p \vee (q \vee q') = p \vee (q' \vee q) = (p \vee q') \vee q \succ p \vee q'.
    \end{align} \phantom{\qedhere}
\end{proof}

\noindent This lemma is all we need to prove the monotonicity of $E^+$ under bistochastic degradation of $q$.

\begin{theorem} \label{th:monotone_future_q}
    There exists no bistochastic matrix $D$ such that $E^- (p \parallel Dq) > E^- (p \parallel q)$.
\end{theorem}

\begin{proof}
    Let us show that the maximum value of $E^+(p \parallel q')$ (with $q' \prec q$) is achieved for $q'= q$.
    \begin{align}
        \max_{q' \prec q} E^- (p \parallel q') &= \max_{q' \prec q} d(p, p \vee q')\\
        &= \max_{q' \prec q} H(p) - H(p \vee q')\\
        \overset{\text{Lemma \ref{lem:monotone_future_q}}}&{=} H(p) - H(p \vee q)\\
        &= d(p, p \vee q)\\
        &= E^- (p \parallel q). \qedhere
    \end{align}
\end{proof}




\section{Discussion}

