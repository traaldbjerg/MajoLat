\chapter{Quantification of incomparability} \label{chap:incomparability}

This chapter tries to quantify the amount of incomparability between probability vectors. The idea of proceeding this way came from chapter \ref{chap:criteria}, where incomparability seemed to allow us to improve entropic criteria. As such, studying how incomparable two distributions are seemed an interesting prospect. It is interesting to note that some unpublished work attempting to quantify the incomparability of quantum states does exist (but without the lattice), however the proposed measures do not seem very natural \cite{hu_characterizing_2018}.



\section{Entropic distance approach}

\subsection{Resource-theoretic intuition}

The approach that was taken was to take inspiration from resource theories to define incomparability monotones. What is often done in resource theories is to define a monotone as being the minimum distance from the studied state to the set of free states. For instance, the entanglement monotone $H(\lambda_\psi)$ is such a monotone. Consider the entropic distance (cf. section \ref{sec:entropic_distances}) $d(\lambda_\psi, \lambda_{\text{free}})$. We know that all minimally entangled states have $\lambda_\text{free} = (1, 0, \dots, 0)$ (they are separable), and so the entropic distance reduces to

\begin{align}
    d(\lambda_\psi, \lambda_\text{free}) &= H(\lambda_\psi) + H(\lambda_\text{free}) - 2 H(\lambda_\psi \vee \lambda_\text{free})\\
                                      &= H(\lambda_\psi),
\end{align}

\noindent where $H(\lambda_\text{free}) = 0$ (the certain distribution has no entropy) and $\lambda_\psi \vee \lambda_\text{free} = \lambda_\text{free}$ because the certain distribution is the bottom of the lattice. In this particular case, there is no need to take the minimum over all free states because they all have the same Schmidt vector, but in other resource theories or using other notions of distance between quantum states, they might not all be at the same distance from the state $\lambda_\psi$.

\subsection{Expected properties}

There are some key differences to usual resource monotones when quantifying incomparability. The first major difference is that a vector can only be incomparable to another vector (a vector is not incomparable in and of itself). As such, treating incomparability as a resource is a bit strange because it would characterize some form of collective property of a pair of states, rather than an intrinsic property of individual states which most resource theories do. In the following, we will call $p$ the \textit{probe state}, and $q$ the \textit{reference state} that $p$ is compared to. Of course, nothing prevents us from reversing the roles of $p$ and $q$ to study the other half of the collective property. 

Let $E$ be some incomparability function. To emphasize the asymmetrical role of $p$ and $q$, we propose the notation $E(p \parallel q)$, which is inspired by the notation for the \textit{relative entropy} (also known as Kullback-Leibler divergence, cf. appendix \ref{app:shannon_compositions}) between 2 probability distributions, which is also asymmetric \cite[p. 19]{cover_elements_2006}. A good measure of incomparability $E$ between $p$ and $q$ should be 0 when $p \sim q$ and greater than 0 only when $p \nsim q$. That is,

\begin{equation} % peut etre moyen d'expliquer ceci un peu mieux en disant que 0 sinon ?
    E(p \parallel q) \neq 0 \iff p \in \mathcal{T}_\emptyset(q). \label{eq:ideal_incomparability}
\end{equation}

Unfortunately, the second difference with usual resource monotones is that the set of comparable vectors to $q$ is not convex. Figure \ref{fig:convex_mixture_incomp} illustrates why this is the case: a convex mixture of a vector from $\mathcal{T}_+(q)$ and $\mathcal{T}_-(q)$ could land in $\mathcal{T}_\emptyset(q)$. % ?? check le souci enft je vois pas exactement ce qui va pas
This problem unfortunately prevents us from achieving a monotone obeying equation (\ref{eq:ideal_incomparability}), at least directly. However, we can define two separate notions of incomparability, one as the minimal distance to the future cone of $q$ and the other as the minimal distance to the past cone of $q$.

\begin{figure}[h!]
    \centering
    \begin{tikzpicture}[scale=0.9]
        % draw cone of q
        \coordinate (A) at (-2,3);
        \coordinate (B) at (2, -3);
        \draw [name path=A--B] (A) -- (B);
        \coordinate (C) at (-2,-3);
        \coordinate (D) at (2,3);
        \draw [name path=C--D] (C) -- (D);
        \path [name intersections={of=A--B and C--D,by=E}];
        \node [fill=black,inner sep=1pt,label=0:$q$] at (E) {};
        % define q^- and q^+
        \coordinate (F) at (1, 2);
        \coordinate (G) at (1, -2);
        \node [fill=black,inner sep=1pt,label=90:$q^-$] at (F) {};
        \node [fill=black,inner sep=1pt,label=270:$q^+$] at (G) {};
        % draw their convex mixtures
        \draw[dotted] [name path=F--G, color=gray] (F) -- (G);
        % fill draw the future cone of p + notation
        \fill[fill=red, opacity=0.2] (C) -- (E) -- (B) -- cycle;
        \node [inner sep=0pt, label=-90:$\mathcal{T}_+ (q)$] at (0, -2) {};
        % fill draw the past cone of p + notation
        \fill[fill=blue, opacity=0.2] (A) -- (E) -- (D) -- cycle;
        \node [inner sep=0pt, label=90:$\mathcal{T}_- (q)$] at (0, 2) {};
        \node [inner sep=0pt, label=180:$\mathcal{T}_\emptyset (q)$] at (-1.5, 0) {};

    
    \end{tikzpicture}
    \caption{Depiction of the set $\mathcal{T}_+(q) \cup \mathcal{T}_-(q)$ not being convex, as there exists elements $q^+, q^-$ such that the chord joining them is not entirely contained in the set.}
    \label{fig:convex_mixture_incomp}
\end{figure}

\subsection{Future incomparability function}

Considering the previous discussion, there are 2 notions of incomparability we can define. The first will be called the notion of \textit{future incomparability}. Essentially, it is simply going to characterize how far $p$ is from the future cone of $q$. We propose the following definition for a future incomaparability function $E^+(p \parallel q)$, illustrated by figure \ref{fig:future_closest_intuition} directly on the lattice.

\begin{figure}[h!]
    \centering
    \begin{tikzpicture}[scale=0.9]
        % draw cone of q
        \coordinate (A) at (-2,3);
        \coordinate (B) at (2, -3);
        \draw [name path=A--B] (A) -- (B);
        \coordinate (C) at (-2,-3);
        \coordinate (D) at (2,3);
        \draw [name path=C--D] (C) -- (D);
        \path [name intersections={of=A--B and C--D,by=E}];
        \node [fill=black,inner sep=1pt,label=0:$q$] at (E) {};
        % ghost draw cone of p
        \coordinate (F) at (0.666,3);
        \coordinate (G) at (-3.333, -3);
        \draw [name path=F--G, draw=none] (F) -- (G);
        \coordinate (H) at (-0.666,-3);
        \coordinate (I) at (-4.666,3);
        \draw [name path=H--I, draw=none] (H) -- (I);
        \path [name intersections={of=F--G and H--I,by=J}];
        \node [fill=black,inner sep=1pt,label=180:$p$] at (J) {};
        \path [name intersections={of=F--G and A--B,by=K}];
        \path [name intersections={of=H--I and C--D,by=L}];
        %\node [fill=black,inner sep=2pt,label=180:closest past state?] at (K) {};
        \node [fill=black,inner sep=1.5pt,label=180:closest future state] at (L) {};
        %\draw[dotted] [name path=J--K, color=gray] (J) -- (K);
        \draw[dotted] [name path=J--L, color=gray] (J) -- (L) node[midway, above, sloped] {$E^+(p \parallel q)$};
        % fill draw the future cone of p + notation
        \fill[fill=red, opacity=0.2] (C) -- (E) -- (B) -- cycle;
        \node [inner sep=0pt, label=-90:$\mathcal{T}_+ (q)$] at (0, -2) {};
        % fill draw the past cone of p + notation
        %\fill[fill=blue, opacity=0.2] (A) -- (E) -- (D) -- cycle;
        %\node [inner sep=0pt, label=90:$\mathcal{T}_- (q)$] at (0, 2) {};
        % fill draw the incomparable region of p mais pas hyper clair
        %\filldraw[draw=black, fill=gray, opacity=0.2] (A) -- (E) -- (C) -- cycle;
        %\filldraw[draw=black, fill=gray, opacity=0.2] (D) -- (E) -- (B) -- cycle;
        %\node [inner sep=0pt, label=0:$\mathcal{T}_\emptyset (q)$] at (1.5, 0) {};
    
    \end{tikzpicture}
    \caption{Depiction of the geometrical intuition behind the closest 'free state' approach.}
    \label{fig:future_closest_intuition}
\end{figure}

\begin{definition}[Future incomparability function] \label{def:future_incomparability_monotone}
    Let $p, q \in \mathcal{P}^d$. The future incomparability $E^+(p \parallel q)$ of $p$ to $q$ is defined as
    \begin{equation}
        E^+ (p \parallel q) = \min_{s \succ q} d(p, s).
    \end{equation}
\end{definition}

\begin{remark}
    From the definition it is easy to see that $p \in \mathcal{T}_+(q) \iff E^+(p \parallel q) = 0$, because then the closest state to $p$ that also majorizes $q$ is simply $p$.
\end{remark}

Essentially, we are looking for the vector $s$ in $\mathcal{T}_+(q)$ which is closest to $p$ in terms of entropic distance. Intuitively, we would expect this vector to be $p \vee q$. Lemma \ref{lem:comp_future} gives us the tools to prove the first major result of this chapter, which essentially states that our geometric intuition is correct.

\begin{theorem}
    The future incomparability of a vector $p$ to another vector $q$ is the entropic distance from $p$ to their join. Formally,
    \begin{equation}
        E^+ (p \parallel q) = d(p, p \vee q).
    \end{equation}
\end{theorem}

Essentially, this theorem simply states that our geometric intution concerning majorization cones is correct (in this case). Let us now prove this result.

\begin{proof}
    Let us find the minimal value of $d(p, s)$ (with $s \succ q$), and let us show that it is realized for $s = p \vee q$.

    \begin{align}
        E^+(p \parallel q) &= \min_{s \succ q} d(p, s) \\
        \overset{\text{Lemma \ref{lem:comp_future}}}&{=} \min_{s \succ q} d(p, p \vee s) + d(p \vee s, s)
    \end{align}

    \noindent Let $s' = p \vee s$. The absorption law $a \vee a = a$, along with the associativity of the join operation, implies that choosing $s = s'$ (i.e. requiring that $s$ majorizes $p$) makes the second term of the sum vanish, while the first remains invariant.

    \begin{align}
        \implies E^+ (p \parallel q) &= \min_{s \succ q} d(p, p \vee s') + d(p \vee s', s') \\
        &= \min_{s \succ q} d(p, p \vee (p \vee s)) + d(p \vee (p \vee s), p \vee s) \\
        &= \min_{s \succ q} d(p, (p \vee p) \vee s) + d((p \vee p) \vee s, p \vee s)\\
        &= \min_{s \succ q} d(p, p \vee s) + d(p \vee s, p \vee s)\\
        &= \min_{s \succ q} d(p, p \vee s)
    \end{align}

    \noindent Because $s$ majorizes $q$, $s'$ majorizes $q$ too. Moreover, to be the join of $p$ with another vector, $s'$ must majorize $p$ as well.

    \begin{align}
        \implies E^+ (p \parallel q) &= \min_{s' \succ q, p} d(p, s') \\
        &= \min_{s' \succ q, p} H(p) + H(s') - 2H(p \vee s') \\
        &= \min_{s' \succ q, p} H(p) + H(s') - 2H(s') \\
        &= \min_{s' \succ q, p} H(p) - H(s').\\
    \end{align}

    \noindent The Schur-concavity of the Shannon entropy implies that the maximum value of $H(s')$ for $s' \in \{v \mid v \succ p, q\}$\footnote{This is equivalent to saying $s' \in \mathcal{T}_+(p) \cap \mathcal{T}_+(q)$.}$ \subseteq \mathcal{P}^d$ is reached for a vector $s'$ that is majorized by all other vectors in the subset. By definition, that vector is the join $p \vee q$. \qedhere
\end{proof}



\subsection{Past incomparability function}

One could attempt the same definition for a \textit{past incomparability} function of $p$ to $q$ as being the minimal distance from $p$ to the past cone of $q$, $\mathcal{T}_-(q)$. However, using the same entropic distance does not yield properties as nice as we had with the join. A simple interpretation of this would be the hyperbolic geometry induced by our entropic distance, which complicates any attempts at working with the meet. Instead, we worked with the entropic quasidistance $d'(p, q) = 2H(p \wedge q) - H(p) - H(q)$ as introduced in definition \ref{def:entropic_quasidistance}. We propose the following definition for a past incomparability function $E^-(p \parallel q)$, illustrated by figure \ref{fig:past_closest_intuition}.

\begin{figure}[h!] % rajouter les 4 distances sur la figure ?
    \centering
    \begin{tikzpicture}[scale=0.9]
        % draw cone of q
        \coordinate (A) at (-2,3);
        \coordinate (B) at (2, -3);
        \draw [name path=A--B] (A) -- (B);
        \coordinate (C) at (-2,-3);
        \coordinate (D) at (2,3);
        \draw [name path=C--D] (C) -- (D);
        \path [name intersections={of=A--B and C--D,by=E}];
        \node [fill=black,inner sep=1pt,label=0:$q$] at (E) {};
        % ghost draw cone of p
        \coordinate (F) at (0.666,3);
        \coordinate (G) at (-3.333, -3);
        \draw [name path=F--G, draw=none] (F) -- (G);
        \coordinate (H) at (-0.666,-3);
        \coordinate (I) at (-4.666,3);
        \draw [name path=H--I, draw=none] (H) -- (I);
        \path [name intersections={of=F--G and H--I,by=J}];
        \node [fill=black,inner sep=1pt,label=180:$p$] at (J) {};
        \path [name intersections={of=F--G and A--B,by=K}];
        %\path [name intersections={of=H--I and C--D,by=L}];
        \node [fill=black,inner sep=1.5pt,label=180:closest past state] at (K) {};
        %\node [fill=black,inner sep=2pt,label=180:closest future state?] at (L) {};
        \draw[dotted] [name path=J--K, color=gray] (J) -- (K) node[midway, below, sloped] {$E^-(p \parallel q)$};
        %\draw[dotted] [name path=J--L, color=gray] (J) -- (L);
        % fill draw the future cone of p + notation
        %\fill[fill=red, opacity=0.2] (C) -- (E) -- (B) -- cycle;
        %\node [inner sep=0pt, label=-90:$\mathcal{T}_+ (q)$] at (0, -2) {};
        % fill draw the past cone of p + notation
        \fill[fill=blue, opacity=0.2] (A) -- (E) -- (D) -- cycle;
        \node [inner sep=0pt, label=90:$\mathcal{T}_- (q)$] at (0, 2) {};
        % fill draw the incomparable region of p mais pas hyper clair
        %\filldraw[draw=black, fill=gray, opacity=0.2] (A) -- (E) -- (C) -- cycle;
        %\filldraw[draw=black, fill=gray, opacity=0.2] (D) -- (E) -- (B) -- cycle;
        %\node [inner sep=0pt, label=0:$\mathcal{T}_\emptyset (q)$] at (1.5, 0) {};
    
    \end{tikzpicture}
    \caption{Depiction of the geometrical intuition behind the closest 'free state' approach.}
    \label{fig:past_closest_intuition}
\end{figure}

\begin{definition}[Past incomparability function]
    Let $p, q \in \mathcal{P}^d$. The past incomparability $E^-(p \parallel q)$ of $p$ to $q$ is defined as
    \begin{equation}
        E^- (p \parallel q) = \min_{s \prec q} d'(p, s).
    \end{equation}
\end{definition}

The same interpretation can be given as the future incomparability function, though one should be careful that we are working with a quasidistance this time around, and so geometric intuitions can be deceiving.

\begin{remark}
    From the definition it is easy to see that $p \in \mathcal{T}_-(q) \iff E^-(p \parallel q) = 0$, because then the closest state to $p$ that is also majorized by $q$ is simply $p$.
\end{remark}

With this quasidistance, we fall on similar properties as we did with the future incomparability monotone. Namely, the minimal quasidistance to the past cone is reached for the meet $p \wedge q$. Before proving this, we will need the following lemma, very similar to lemma \ref{lem:comp_future} except that this time around we can't speak of triangular inequality because of the quasidistance.

\begin{lemma} \label{lem:comp_past}
    Let $p, q \in \mathcal{P}^d$. For the entropic quasidistance $d'(p, q) = 2H(p \wedge q) - H(p) - H(q)$, we have
    \begin{equation}
        d'(p, q) = d'(p, p \wedge q) + d'(p \wedge q, q).
    \end{equation}
\end{lemma}

\begin{proof}
    \begin{align}
        d'(p, p \wedge q) + d'(p \wedge q, q) &= 2H(p \wedge (p \wedge q)) - H(p) - H(p \wedge q) + 2H((p \wedge q) \wedge q) \nonumber\\
        &\quad \quad - H(p \wedge q) - H(q)\\
        &= 2H(p \wedge q) - H(p \wedge q) - H(p) + 2H(p \wedge q) - H(p \wedge q)\nonumber\\
        &\quad \quad - H(q) \\
        &= 2H(p \wedge q) - H(p) - H(q) \\
        &= d'(p, q). \qedhere
    \end{align} 
\end{proof}

We are now ready to prove the main result of this section, which again states that our geometric intuition is correct (though this time the interpretation is murkier given the quasidistance).

\begin{theorem}
    The past incomparability of a vector $p$ to another vector $q$ is the entropic distance\footnote{While the definition made no mention of the rigorous entropic distance $d$, this theorem brings us back to the actual distance.} from $p$ to their meet. Formally,
    \begin{equation}
        E^- (p \parallel q) = d(p, p \wedge q).
    \end{equation}
\end{theorem}

\begin{proof}
    Let us find the minimal value of $d'(p, s)$ (with $s \prec q$), and let us show it is realized for $s = p \wedge q$.
    \begin{align}
        E^- (p \parallel q) &= \min_{s \prec q} d'(p, s) \\
        \overset{\text{Lemma \ref{lem:comp_past}}}&{=} \min_{s \prec q} d'(p, p \wedge s) + d'(p \wedge s, s) \\ 
        &\implies \min_{s \prec q} d'(p, s) \geq \min_{s \prec q} d'(p, p \wedge s) \label{eq:1}
    \end{align}

    \noindent Let $s' = p \wedge s$. Because $s$ is majorized by $q$, $s'$ is majorized by $q$ as well. Moreover, to be the meet of $p$ with another vector, $s'$ must be majorized by $p$ as well.

    \begin{align}
        \implies \min_{s \prec q} d'(p, p \wedge s) &= \min_{s' \prec q, p} d'(p, s')\\
        &= \min_{s' \prec q, p} 2H(p \wedge s') - H(p) - H(s')\\
        &= \min_{s' \prec q, p} H(s') - H(p).
    \end{align}

    \noindent The Schur-concavity of the Shannon entropy implies that the minimum value of $H(s')$ for  $s' \in \{v \mid v \prec p, q\}$\footnote{This is equivalent to saying $s' \in \mathcal{T}_-(p) \cap \mathcal{T}_-(q)$.}$ \subseteq \mathcal{P}^d$ is reached for a vector $s'$ that majorizes all other vectors in the subset. By definition, that vector is the meet $p \wedge q$, and so $\min_{s' \prec q, p} H(s') - H(p) = H(p \wedge q) - H(p)$. The vector $p \wedge q$ is also part of the original subset $\{v \mid v \prec q\}$, and so plugging $s = p \wedge q$ into (\ref{eq:1}) shows that the LHS realizes the value $H(p \wedge q) - H(p)$ as well, and the 2 minima must therefore be equal. Moreover, at the final line we have $H(s') - H(p) = d'(s', p) = d(s', p)$, because the two distance notions are equal when vectors are comparable. \qedhere
\end{proof}



\section{Properties}

For this section, we will denote bistochastic matrices of dimension $d\times d$ by the letter $D$. Recall that an equivalent definition of majorization is $p \prec q \iff \exists D \: | \: q = Dp$ (cf. section \ref{sec:bistochastic}). Now that we have defined our two incomparability functions and shown that they are equal to the (quasi)distance to the meet or join, let us show that they are monotones under a bistochastic degradation. Conversely, if they are increasing (resp. decreasing) monotones under a bistochastic degradation, they are a decreasing (resp. increasing) monotone under a LOCC degradation in the quantum picture. However, we are working with 2 states, and so we can study degradations of $p$ and of $q$ separately.



\subsection{Monotonicity under bistochastic degradation of the probe} \label{sec:p_monotonicity}

Let us start with studying the future incomparability function. If we can show that $E^+(Dp \parallel q)$ is greater than $E^+(p \parallel q)$ for any bistochastic matrix $D$ (implying $Dp \in \mathcal{T}_-(p)$), then we can promote our future incomparability function $E^+$ to an increasing monotone under bistochastic degradation of $p$, or alternatively to a decreasing monotone under LOCC degradation. Figure \ref{fig:future_p_monotonicity} shows why we geometrically expect this to be the case. The representation seems to indicate that the states with the lowest distance to $\mathcal{T}_+(q)$ are on the edge facing towards $q$. All of the states on the edge visually have the same distance to the cone, like $p'$. But, remembering the hyperbolic metric induced by the entropic distance (cf. section \ref{sec:entropic_distances}), we deduce that $d(p', p' \vee q) \leq d(p, p \vee q)$.

\begin{figure}[h!] % rajouter les 4 distances sur la figure ?
    \centering
    \begin{tikzpicture}[scale=0.9]
        % draw cone of q
        \coordinate (A) at (0, 3);
        \coordinate (B) at (-2,0);
        \coordinate (C) at (2, 0);
        \draw [name path=A--B] (A) -- (B);
        \draw [name path=A--C] (A) -- (C);
        \node [fill=black,inner sep=1pt,label=0:$q$] at (A) {};
        % drawfill future cone of q
        \fill[fill=red, opacity=0.2] (A) -- (B) -- (C) -- cycle;
        \node [inner sep=0pt, label=-90:$\mathcal{T}_+ (q)$] at (0, 1.5) {};
        % draw cone of p
        \coordinate (D) at (-2, 1.5);
        \coordinate (E) at (-4, 4.5);
        \coordinate (F) at (0, 4.5);
        \draw[dotted] [name path=D--E] (D) -- (E);
        \draw[dotted] [name path=D--F] (D) -- (F);
        \node [fill=black,inner sep=1pt,label=225:$p$] at (D) {};
        % draw bottom right leg for p
        \coordinate (K) at (-1, 0);
        \draw[draw=none] [name path=D--K] (D) -- (K);
        \path [name intersections={of=D--K and A--B, by=L}];
        \draw[dotted] [name path=D--L, color=gray] (D) -- (L);
        % add p' and p''
        \coordinate (P') at (-1, 3);
        \coordinate (P'') at (-2.5, 3.5);
        \node [fill=black,inner sep=1pt,label=90:$p'$] at (P') {};
        \node [fill=black,inner sep=1pt,label=145:$p''$] at (P'') {};
        % ghost draw bottom right cone leg for p' and p''
        \coordinate (G) at (1, 0);
        \coordinate (H) at (-0.16666, 0);
        \draw[draw=none] [name path=P'--G] (P') -- (G);
        \draw[draw=none] [name path=P''--H] (P'') -- (H);
        \path [name intersections={of=P'--G and A--B, by=I}];
        \path [name intersections={of=P''--H and A--B, by=J}];
        \draw[dotted] [name path=P'--I, color=gray] (P') -- (I);
        \draw[dotted] [name path=P''--J, color=gray] (P'') -- (J);

    \end{tikzpicture}
    \caption{Depiction of the expected monotonicity in $p$ of $E^+$.}
    \label{fig:future_p_monotonicity}
\end{figure}

\noindent In order to prove monotonicity, we will first need the following lemma.

\begin{lemma} \label{lem:incomparable_diamond}
    In any incomparable diamond $p, q, p \wedge q, p \vee q$ we have $d(p, p \vee q) \leq d(q, p \wedge q)$.
\end{lemma}

\begin{proof}
    Let $p, q \in \mathcal{P}^d$. We have
    \begin{align}
        d(p, p \vee q) &= H(p) + H(p \vee q) - 2H(p \vee (p \vee q))\\
        &= H(p) - H(p \vee q)\\
        &= H(p) + H(p \wedge q) - H(p \wedge q) - H(p \vee q)\\
        \overset{\text{supermod}}&{\leq} H(p \wedge q) - H(q)\\
        &= d(q, p \wedge q). \qedhere
    \end{align} 
\end{proof}

This lemma is essentially just an equivalent way of stating supermodularity. We are now ready to prove the main theorem of this section.

\begin{theorem} \label{th:monotone_future_p}
    There exists no bistochastic matrix $D$ such that $E^+ (Dp \parallel q) < E^+ (p \parallel q)$.
\end{theorem}

This theorem essentially states that $E^+$ is an increasing monotone under bistochastic degradation of $p$. This proof is perhaps the trickiest of the chapter, so figure FAIRE FIGURE illustrates the different vectors of the construction on the lattice to help with comprehension.

\begin{proof}
    Let us show that the minimal value of $E^+ (p' \parallel q)$ (with $p' \prec p$) is realized for $p' = p$. Let $p'$ be a vector majorized by $p$, i.e. there exists a bistochastic matrix $D$ such that $p' = Dp$. Let $q'(p') = p' \vee q$, and let $p''(p') = p \wedge q'$ which are both functions of the variable to minimize over, $p'$. To avoid cluttering the expressions, we will simply write $p''$ and $q'$, but one should keep in mind that they are indeed functions of $p'$. By hypothesis and by definition of $q'$, $p'$ is majorized by both $p$ and $q'$, which is equivalent to $p'$ being majorized by $p \wedge q'$. We have

    \begin{equation}
         d(p'', q') = H(p \wedge q') - H(q') \leq H(p') - H(q') = d(p', q'), \label{eq:construction}
    \end{equation} 

    \noindent which can be used for the following development

    \begin{align}
        \min_{p' \prec p} E^+ (p' \parallel q) &= \min_{p' \prec p} d(p', p' \vee q)\\
        &= \min_{p' \prec p} d(p', q')\\
        \overset{\text{(\ref{eq:construction})}}&{=} \min_{p' \prec p} d(p'', q')\\
        &= \min_{p' \prec p} d(p \wedge q', q')\\
        \overset{\text{Lemma \ref{lem:incomparable_diamond}}}&{\geq} \min_{p' \prec p} d(p, p \vee q')\\
        &= \min_{p' \prec p} d(p, p \vee (p' \vee q))\\
        &= \min_{p' \prec p} d(p, (p \vee p') \vee q)\\
        &= d(p, p \vee q)\\
        &= E^+ (p \parallel q),
    \end{align}
    \noindent and so the minimal value of $E^+ (Dp \parallel q)$ is reached for the identity degradation which leaves $p$ invariant. \qedhere
\end{proof}
 
This theorem is quite satisfying in the sense that it seems to indicate that our geometric intuitions on the lattice are valid. Moreover, we can now promote the future incomparability function to a future incomparability \textit{monotone}. Theorem \ref{th:monotone_future_p} has a few corollaries.

\begin{corollary} \label{cor:incomparability_LOCC}
    Let $\ket{\psi}$ and $\ket{\phi}$ be two pure quantum states, and let $\lambda_\psi$ and $\lambda_\phi$ be the associated Schmidt vectors. If $\ket{\psi} \overset{\text{LOCC}}{\longrightarrow} \ket{\phi}$ with probability 1, then $E^+ (\lambda_\psi \parallel \lambda_\alpha) \geq E^+ (\lambda_\phi \parallel \lambda_\alpha)$ with $\lambda_\alpha$ some probability vector.
\end{corollary}

\begin{corollary} \label{cor:incomparability_separability}
    Let $\rho_{AB}$ be a bipartite quantum state, and let $\rho_A$ and $\rho_B$ be the reduced states. Let $\lambda_{AB}, \lambda_A$ and $\lambda_B$ be the vectors of eigenvalues of their density matrices. Then, if $\rho_{AB}$ is separable, $E^+ (\lambda_{AB} \parallel \lambda_B) \geq E^+ (\lambda_A \parallel \lambda_B)$ and $E^+ (\lambda_{AB} \parallel \lambda_A) \geq E^+ (\lambda_B \parallel \lambda_A)$.
\end{corollary}

These corollaries, while not very useful per se, still hold some interpretational value, because they show that in the sense of our future incomparability monotone, some states must be more incomparable than others relatively to some set states. Corollary \ref{cor:incomparability_LOCC}, while involving some nondescript Schmidt vector $\lambda_\alpha$, essentially states it is not possible to create a state that is more incomparable to $\lambda_\alpha$ through LOCC than the original state $\lambda_\psi$. Corollary \ref{cor:incomparability_separability} is perhaps more interesting, and states that if a joint state is separable, then the joint state is more incomparable to each of the reduced states than the reduced states are to each other.

One would hope that the analogue of theorem \ref{th:monotone_future_p} would hold for the measure of past incomparability $E^-$, however it does not. The following counterexample shows that a bistochastic matrix $D$ such that $E^- (Dp \parallel q) < E^- (p \parallel q)$ can exist, but also that a bistochastic matrix $D'$ such that $E^- (D'p \parallel q) > E^- (p \parallel q)$ can exist too. Let $q = (0.6, 0.4)$, $p = (0.7, 0.29, 0.01)$, $p' = p \wedge q$ and $p'' = (0.7, 0.15, 0.15)$. One can verify that $p$ majorizes both $p'$ and $p''$, yet $E^- (p \parallel q) = 0.0939$ bits, $E^- (p' \parallel q) = 0$ bits and $E^- (p'' \parallel q) = 0.171$ bits. With figure FAIRE FIGURE it is not too hard to see that this is because the hyperbolic geometry makes distance in the wrong way in this case: sticking to the right edge for $p'$ makes $E^-$ increase, whereas with $p''$ one can go straight towards the past cone of $q$, making $E^-$ decrease, and so there is no monotonicity.



\subsection{Monotonicity under bistochastic degradation of the reference} \label{sec:q_monotonicity}

Let us now turn our attention to bistochastic degradations of $q$, and attempt to promote our incomparability functions to monotones. Figure \ref{fig:future_q_monotonicity} shows the monotonicity relations we expect. The proofs of these properties are simpler than for probe degradation, and can be found in appendix \ref{app:q_monotonicity}. Starting with the future incomparability function, if we can show that $E^+(p \parallel Dq) < E^+(p \parallel q)$ for any bistochastic matrix $D$, then $E^+$ would be a decreasing monotone, like we expect.

\begin{figure}[h!] % rajouter les 4 distances sur la figure ?
    \centering
    \begin{tikzpicture}[scale=0.9]
        % draw cone of q
        \coordinate (Q) at (0, 3);
        \coordinate (B) at (-2,0);
        \coordinate (C) at (2, 0);
        \draw[dotted] [name path=Q--B, color=gray] (Q) -- (B);
        \draw[dotted] [name path=Q--C, color=gray] (Q) -- (C);
        \node [fill=black,inner sep=1pt,label=0:$q$] at (Q) {};
        \coordinate (E) at (-2, 6);
        \coordinate (F) at (2, 6);
        \draw [name path=Q--E] (Q) -- (E);
        \draw [name path=Q--F] (Q) -- (F);
        % draw cone of p
        \coordinate (P) at (-4, 3);
        \node [fill=black,inner sep=1pt,label=0:$p$] at (P) {};
        \coordinate (D) at (-2.5, 0);
        % add q' and q''
        \coordinate (Q') at (1, 4.5);
        \coordinate (Q'') at (-1, 4.5);
        \node [fill=black,inner sep=1pt,label=0:$q'$] at (Q') {};
        \node [fill=black,inner sep=1pt,label=70:$q''$] at (Q'') {};

    \end{tikzpicture}
    \caption{Depiction of the expected monotonicity in $q$ of $E^+$ and $E^-$.}
    \label{fig:future_q_monotonicity}
\end{figure}

\begin{theorem} \label{th:monotone_future_q}
    There exists no bistochastic matrix $D$ such that $E^+ (p \parallel Dq) > E^+ (p \parallel q)$.
\end{theorem}

Let us turn our attention to $E^-$. This time around, the expected property for $E^-$ does hold, and we do have monotonicity in $q$. If we can show that $E^-(p \parallel Dq) > E^+(p \parallel q)$ for any bistochastic matrix $D$, then $E^-$ would be an increasing monotone, like we expect.

\begin{theorem} \label{th:monotone_past_q}
    There exists no bistochastic matrix $D$ such that $E^- (p \parallel Dq) < E^- (p \parallel q)$.
\end{theorem}

To conclude this section, we will simply note that theorems \ref{th:monotone_future_q} and \ref{th:monotone_past_q} have similar corollaries to those of theorem \ref{th:monotone_future_p}.



\subsection{Compositions}

Now that we have these monotones and have proved some monotonicity properties, we could attempt to build compositions of the two that would have some nice properties. We propose the following definitions.

\begin{definition}[Distance-like incomparability function] \label{def:distance-like_function}
    Let $p, q \in \mathcal{P}^d$. The distance-like incomparability function $F$ is defined as 
    \begin{equation}
        F(p \parallel q) = E^+(p \parallel q) - E^-(p \parallel q).
    \end{equation}
\end{definition}

\begin{definition}[Area-like incomparability function] \label{def:area-like_function}
    Let $p, q \in \mathcal{P}^d$. The area-like incomparability function $G$ is defined as 
    \begin{equation}
        G(p \parallel q) = E^+(p \parallel q) E^-(p \parallel q).
    \end{equation}
\end{definition}

Both of these definitions have interesting properties that might be desirable depending on different use cases. For $F$, a nice property is that theorems \ref{th:monotone_future_q} and \ref{th:monotone_past_q} guarantee that $F$ is a decreasing monotone under bistochastic degradation of $q$. However, $F$ is not necessarily positive. It is unknown whether it is a monotone under bistochastic degradation of $p$.

For $G$, the advantage is that it is positive, and it does satisfy equation (\ref{eq:ideal_incomparability}), because $E^+(p \parallel q) = 0 \iff p \in \mathcal{T}_+(q)$ and $E^-(p \parallel q) = 0 \iff p \in \mathcal{T}_-(q)$. The only region where $G(p \parallel q) \neq 0$ is $\mathcal{T}_\emptyset(q)$, which is a very nice property for a measure of incomparability. However, $G$ loses the monotonicity under bistochastic degradation of $q$. It is unknown whether it is a monotone under bistochastic degradation of $p$.

\section{Discussion}

We will voluntarily keep the discussion in this section short because a better interpretation of $E^+$ and $E^-$ will be made possible with the properties of Shannon entropy we uncover in chapter \ref{chap:volume}. We also postpone discussing applications of those 2 monotones to chapter \ref{chap:volume}. The properties of the distance-like and area-like functions $F$ and $G$ could also be interesting to study, but another direction was chosen. Of the two, $G$ seems like the more promising candidate for a measure of incomparability between two vectors.

While we did take inspiration from a resource-theoretic approach, we have not defined a full resource theory. Indeed, although we have identified free states as being the future-comparable or past-comparable states of $q$, it is not clear what the free operations of the theory might be. Perhaps some more general notion of operation should be defined which would act on both $p$ and $q$, and where $p$ ending in a reachable region for $q$ would make it free because it is already reachable from $q$. This (potential) new type of relative resource theory might be interesting in considering the intrinsic value of having a diverse set of states: if $p \nsim q$, then $p$ can reach some states that $q$ can't (e.g. in a LOCC context). In this sense, incomparability seems like it enables \textit{diversity} in reachable states.

Therefore, incomparability seems like a desirable property, and it is this idea that was explored further. More precisely, now that we have quantified the incomparability between a probe state and a reference state, the question that ended up being interesting to ask was how can we generalize this notion of incomparability to a \textit{set of reference states}?