\chapter{Entropic volume of majorization cones} \label{chap:volume}

This chapter is a direct continuation of chapter \ref{chap:incomparability}. The question that this chapter tried to answer is the following: how can we generalize the notion of incomparability function of a probe state, relative to a \textit{set of reference states} ? For the rest of this master thesis, we will call this set of reference states the \textit{bank}. We will also focus on LOCC-reachability, which are confined to the $\mathcal{T}_+$ sets, and will only discuss the generalization of $E^+$. We did not attempt it, but we believe that reversing the whole discussion of this chapter is also possible for other theories such as quantum thermodynamics, which work with the $\mathcal{T}_-$ sets instead, on the condition to work with a negentropy-based $E^-$.



\section{Entropic volume}

\subsection{Venn diagrams and the inclusion-exclusion principle}

Let us remain in the 2 state context for now. Incomparability seems to enable \textit{diversity} in the set of LOCC-reachable states from $p$ and $q$. In some sense, we would expect $E^+$ to characterize how many states are only accessible from $p$, but not from $q$. The set-theoretic intuition is helpful here: we are looking to compute the volume (which is directly linked to the amount of reachable states) $V(\mathcal{T}_+(p) \backslash \mathcal{T}_+(q))$, where $V$ is some volume-like\footnote{We will touch on what this means a bit later.} set function that sends the cone of a state to a real number. For sets $A$ and $B$, set theory gives us the following expression

\begin{equation} \label{eq:2_set_exclusion}
    V(A \backslash B) = V(A) - V(A \cap B).
\end{equation}

The formula is easily understood with the Venn diagram on figure FAIRE FIGURE. The key insight which we will use extensively for this chapter, which to our knowledge has not been used in the litterature before, is that $\mathcal{T}_+(p) \cap \mathcal{T}_+(q) = \mathcal{T}_+(p \vee q)$ (we had briefly mentioned this relationship with figure \ref{fig:meet_join_example}). It is very important to note at this stage that while visually similar, the notation for the lattice-theoretic meet and join $\wedge$ and $\vee$ should not be confused with the notations for the set-theoretic union and intersection $\cup$ and $\cap$. With this insight, we can rewrite equation (\ref{eq:2_set_exclusion}) with $A = \mathcal{T}_+(p)$ and $B = \mathcal{T}_+(q)$ as

\begin{equation}
    V(\mathcal{T}_+(p) \backslash \mathcal{T}_+(q)) = V(\mathcal{T}_+(p)) - V(\mathcal{T}_+(p \vee q)).
\end{equation}

Thankfully, it is possible to generalize this idea further. For example, for three sets $A, B, C$ we have the statement

\begin{equation}
    V(A \backslash (B \cup C)) = V(A) - V(A \cap B) - V(A \cap C) + V(A \cap B \cap C),
\end{equation}

which is illustrated with Venn diagrams on figure FAIRE FIGURE. The idea is essentially that when removing both $A \cap B$ and $A \cap C$ from $A$, the region $A \cap B \cap C$ gets removed twice, and we must add it back to get the correct set size. This formula can then be rewritten with $A = \mathcal{T}_+(p)$, $B = \mathcal{T}_+(q_1)$ and $C = \mathcal{T}_+(q_2)$ to compute the volume of the region that is \textit{only} LOCC-reachable from $p$, and not from the bank $\{q_1, q_2\}$. The fully general set-theoretic formula is known as the \textit{inclusion-exclusion principle}, which gives the following statement.

\begin{theorem}[Inclusion-exclusion principle CITATION NEEDED] \label{th:inclusion-exclusion}
    Let $A_i \subseteq \Omega, 0 \leq i \leq k$, and let $V$ be some volume-like set function over $\Omega$. We have the following statement
    \begin{equation}
        V\left(A_0 \backslash \left(\bigcup\limits_{i = 1}^k A_i\right)\right) = V(A_0) + \sum_{\emptyset \neq J \subseteq \{1, \dots, k\}} (-1)^{|J|} \: V\left(A_0 \bigcap\limits_{i \in J} A_i\right),
    \end{equation}
    where the sum $\sum_{\emptyset \neq J \subseteq \{1, \dots, k\}}$ denotes the sum over all possible combinations $J$ of elements from $\{1, \dots, k\}$ (excluding the combination with no elements) and where $|J|$ denotes the cardinality of the set $J$.
\end{theorem}

Several equivalent versions of theorem \ref{th:inclusion-exclusion} exist, but this version is the relevant one in our context. We propose the following corollary.

\begin{corollary}[LOCC inclusion-exclusion] \label{corr:LOCC_inclusion-exclusion}
    Let $p, q_1, \dots, q_k \in \mathcal{P}^d$, and let $V$ be some volume-like set function over $\mathcal{P}^d$. Then
    \begin{equation} \label{eq:LOCC_inclusion-exclusion}
        V\left(\mathcal{T}_+(p) \backslash \left(\bigcup\limits_{i = 1}^k \mathcal{T}_+(q_i)\right)\right) = V\left(\mathcal{T}_+(p)\right) + \sum_{\emptyset \neq J \subseteq \{1, \dots, k\}} (-1)^{|J|} \: V\left(\mathcal{T}_+\left(p \bigvee\limits_{i \in J} q_i\right)\right),
    \end{equation}
    where $p \bigvee\limits_{i \in J} q_i$ is defined as $p \vee q_{j_1} \vee \dots \vee q_{j_n}$ (with $\{j_1, \dots, j_n\} = J$), is the volume of the set of states that is only LOCC-reachable from $p$ but not from the bank of states $\{q_1, \dots, q_k\}$.
\end{corollary}

Let us now turn our attention to what it means for a set function to be volume-like. The main defining property is that such a function must be \textit{finitely additive}.

\begin{definition}[Finite additivity CITATION NEEDED]
    A set function $F: \Omega \rightarrow \mathbb{R}$ is finitely additive iff for all sets $A_i \subseteq \Omega, 1 \leq i \leq k$ that are mutually disjoint (i.e. $A_i \cap A_j = \emptyset \: \forall \: i \neq j$), we have
    \begin{equation}
        F\left(\bigcup\limits_{i = 1}^k A_i\right) = \sum_{i=1}^k F(A_i).
    \end{equation}
\end{definition}

In most textbooks, the inclusion-exclusion principle is derived from the property of finite additivity. However, they are equivalent as one implies the other CITATION NEEDED.



\subsection{Generalization of incomparability monotones to a bank of states}

The volume of convex polytopes in Weyl chambers (which our majorization cones are) under a Euclidean measure can be computed explicitly. Unfortunately, no closed-form solution exists past dimension 3, although algorithms do exist to compute them in any dimension \cite{junior_geometric_2022}. However, compute time and memory requirements can quickly become limiting as the dimensions increase, rendering such methods difficult to use \cite{bueler_exact_2000} A REVERIFIER.

There is an additional difficulty in the field of LOCC transformations: quantum states are not distributed evenly on the canonical Weyl chamber of a $\Delta_{d-1}$ simplex. This is because the Schmidt decomposition induces a non-euclidean measure, called a Haar measure (which essentially captures the density of quantum states in the Schmidt vector space). Volume formulae taking this Haar measure into account do exist, and are useful in quantifying things like the average entanglement in a region of density matrices \cite{zyczkowski_induced_2001}.

However, we will not study them further in this master thesis. Indeed, in terms of an entanglement QRT, it is not entirely clear why a Schmidt vector shared by many quantum states should be weighted more than a Schmidt vector that is shared by few quantum states. This is because states with the same Schmidt vector are equivalent up to a local change of basis, and can thus be interchanged without losing any resource. Of course, we are not trying to argue that assigning a non-uniform weight to Schmidt vectors is necessarily useless, but rather that it is not clear why there couldn't exist other relevant choices of weighting than the density of quantum states.

With this in mind, we postulate that Shannon entropy is itself a measure of the volume of majorization cones under some unknown measure. This makes our future incomparability monotone $E^+(p \parallel q) = H(p) - H(p \vee q)$ from chapter \ref{chap:incomparability} much easier to interpret: it simply measures the volume of states that are reachable from $p$, but not from $q$. For this reason, it is perhaps more natural to call $E^+$ \textit{unique entropy}. We propose the following generalization of $E^+$.

\begin{definition}[Unique entropy] \label{def:unique_entropy}
    Let $p, q_1, \dots, q_k \in \mathcal{P}^d$. The unique entropy of $p$ relative to the bank $\{q_1, \dots, q_k\}$, which we note $E^+(p \parallel q_1, \dots, q_k)$, is defined as 
    \begin{equation} \label{eq:unique_entropy}
        E^+(p \parallel q_1, \dots, q_k) = H(p) + \sum_{\emptyset \neq J \subseteq \{1, \dots, k\}} (-1)^{|J|} \: H\left(p \bigvee\limits_{i \in J} q_i\right).
    \end{equation}
\end{definition}

While this definition and postulate may not seem too far-fetched considering the previous discussion, this essentially means that if $E^+$ satisfies volume-like behavior, then a set function $\mu_0$ which sends cones onto the entropy of their tip $p$ is a valid pre-measure of the set $\mathcal{P}^d$. In other words, the volume of a cone would be entirely determined by its tip (meaning that $\mu_0$ would be a very unusual set function). That this might be possible is not too surprising however, because existing formulae for computing the volume of convex polytopes already only use the vertices of the polytope. Moreover, algorithms to compute the location of the vertices of the majorization cone of any vector $p$ do exist, and Ref. \cite{junior_thermalcones_2022} even provides a Mathematica implementation (in the context of thermomajorization\footnote{It is interesting to note at this point that the volume of thermal cones has already been studied in Ref. \cite{junior_geometric_2022}, and that they also give an interpretation of the volume being akin to a resource. Moreover, they also propose a euclidean and Haar volume for entanglement cones. However, they studied the volume of individual cones, and thus did not attempt to characterize the volume uniquely accessible from a state. Finally, we were also interested in studying how the entropic volume might behave.} which is similar). Therefore, the volume of a majorization cone is entirely determined by its tip.

Some work did go into studying the basics of measure theory and defining a rigorous set function which would send cones onto the entropy of their tip, which would then be a valid measure for majorization cones if it could verify the properties of countable additivity and positivity on the $\sigma$-algebra generated by the majorization cones. Such a result would be very interesting, and would give new insight into the nature of entropy on the majorization lattice.

For this to be possible we need our set function to work on a given collection of subsets of $\mathcal{P}^d$ of interest. We are only really interested in the measure of the cones, however a measure must be defined on a $\sigma$-algebra\footnote{Or on a semi-ring, but this did not seem relevant in our context.} CITATION NEEDED. We can generate the $\sigma$-algebra from the true collection of interest: the $\sigma$-algebra of our majorization cones contains all the possible unions, complements and intersections of cones. Intersections are easy because $\mathcal{T}_+(p) \cap \mathcal{T}_+(q) = \mathcal{T}_+(p \vee q)$ (and so the collection of cones directly forms a $\pi$-system), but unions and complements are not necessarily cones and so the naive set function which sends cones on the entropy of their tip is not directly defined for those. It is unfortunately not clear if and how one could extend the naive set function to the $\sigma$-algebra of cones. We believe that a recursive definition for an extended set function $\mu_{\text{ext}}$ sending a set $A$ to $\mu_0(A)$ if $A$ is a cone, and to $\mu_{\text{ext}}(A') +\mu_{\text{ext}}(A'') - \mu_{\text{ext}}(A' \cap A'')$ if $A$ is the union of 2 sets $A'$ and $A''$ might be enough, however proving countable additivity for such a recursive function seemed out of scope for this master thesis and was not studied further.



\subsection{Properties} \label{sec:unique_entropy_properties}

Unfortunately, without a clear set function we cannot work on showing finite additivity, and so we will not show that a $H$-based set function is rigorously a volume. However, we will still attempt to show that our unique entropy function $E^+$ holds properties compatible with the geometrical intuition one could have for such a volume. For any $p, q_1, \dots, q_k, q_{k+1} \in \mathcal{P}^d$, we have

\begin{enumerate}
    \item \textbf{Commutativity:} $E^+(p \parallel q_1, \dots, q_i, \dots, q_j, \dots, q_k) = E^+(p \parallel q_1, \dots, q_j, \dots, q_i, \dots, q_k)$ for any $i \neq j$. \label{prop:commutativity}
    \item \textbf{Empty volume:} $E^+(\overline{1}_d \parallel q_1, \dots, q_k) = 0$. \label{prop:empty}
    \item \textbf{Absorption in $p$:} $E^+(p \parallel q_1, \dots, q_k) = 0$ if $\exists i \leq k \: | \: p \succ q_i$. \label{prop:p_absorption}
    \item \textbf{Absorption in $q$:} $E^+(p \parallel q_1, \dots, q_k, q_{k+1}) = E^+(p \parallel q_1, \dots, q_k)$ if $\exists i \leq k \: | \: q_{k+1} \succ q_i$. \label{prop:q_absorption}
    \item \textbf{Positivity:} $E^+(p \parallel q_1, \dots, q_k) \geq 0$. \label{prop:positivity}
    \item \textbf{Monotonicity in $p$:} $E^+(Dp \parallel q_1, \dots, q_k) \geq E^+(p \parallel q_1, \dots, q_k)$ for any bistochastic matrix $D$. \label{prop:p_monotonicity}
    \item \textbf{Monotonicity in $q$:} $E^+(p \parallel q_1, \dots, Dq_i, \dots, q_k) \leq E^+(p \parallel q_1, \dots, q_i, \dots, q_k) \: \forall i \leq k$,  for any bistochastic matrix $D$. \label{prop:q_monotonicity}
\end{enumerate}

\noindent The key insight needed to prove these properties is given by the following lemma.

\begin{lemma} \label{lem:induction_trick}
    Let $p, q_1, \dots, q_k, q_{k+1} \in \mathcal{P}^d$. Then,
    \begin{equation}
        E^+(p \parallel q_1, \dots, q_k, q_{k+1}) = E^+(p \parallel q_1, \dots, q_k) - E^+(p \vee q_{k+1} \parallel q_1, \dots, q_k).
    \end{equation}
\end{lemma}

\begin{proof}
    Let $p, q_1, \dots, q_k, q_{k+1} \in \mathcal{P}^d$. We have the following A REFORMATER
    \begin{align}
        E^+(p \parallel q_1, \dots, q_k, q_{k+1}) &= H(p) - H(p \vee q_1) - \dots - H(p \vee q_k) - H(p \vee q_{k+1}) \nonumber \\
                                                  &\quad + H(p \vee q_1 \vee q_2) + \dots + H(p \vee q_{k-1} \vee q_k) + H(p \vee q_1 \vee q_{k+1}) \nonumber \\
                                                  &\quad + \dots + H(p \vee q_k \vee q_{k+1}) + \dots\\
                                                  &= \big(H(p) - H(p \vee q_1) - \dots - H(p \vee q_k) + H(p \vee q_1 \vee q_2)\nonumber \\
                                                  &\quad + \dots + H(p \vee q_{k-1} \vee q_k) - \dots \big) \nonumber \\
                                                  &\quad - \big(H(p \vee q_{k+1}) - H(p \vee q_1 \vee q_{k+1}) - \dots - H(p \vee q_k \vee q_{k+1})\nonumber \\
                                                  &\quad + H(p \vee q_1 \vee q_2 \vee q_{k+1}) + \dots + H(p \vee q_{k-1} \vee q_k \vee q_{k+1}) - \dots\big). \label{eq:swapparoo_start}
    \end{align}
    By commutativity of the join, we can push $q_{k+1}$ next to $p$, and so equation (\ref{eq:swapparoo_start}) becomes
    \begin{align}
        E^+(p \parallel q_1, \dots, q_k, q_{k+1}) &= \big(H(p) - H(p \vee q_1) - \dots - H(p \vee q_k) + H(p \vee q_1 \vee q_2)\nonumber \\
                                                  &\quad + \dots + H(p \vee q_{k-1} \vee q_k) - \dots \big) \nonumber \\
                                                  &\quad - \big(H(p \vee q_{k+1}) - H(p \vee q_{k+1} \vee q_1 ) - \dots - H(p \vee q_{k+1} \vee q_k )\nonumber \\
                                                  &\quad + H(p \vee q_{k+1} \vee q_1 \vee q_2 ) + \dots + H(p\vee q_{k+1} \vee q_{k-1} \vee q_k) - \dots\big). \label{eq:swapparoo_done}
    \end{align}
    Comparing terms one by one, we can see that the second half of the expression is exactly the same as the first half (which is equal to $E^+(p \parallel q_1, \dots, q_k)$) if $p$ is replaced by $p \vee q_{k+1}$, and so equation (\ref{eq:swapparoo_done}) becomes
    \begin{equation}
        E^+(p \parallel q_1, \dots, q_k, q_{k+1}) = E^+(p \parallel q_1, \dots, q_k) - E^+(p \vee q_{k+1} \parallel q_1, \dots, q_k). \qedhere
    \end{equation}
\end{proof}

Such an expression lends itself very well to induction proofs. Let us prove the different properties. It turns out that only properties \ref{prop:positivity}, \ref{prop:p_monotonicity} and \ref{prop:q_monotonicity} are difficult to prove. The other proofs come fairly naturally from definition \ref{def:unique_entropy}, and can be found in appendix \ref{app:unique_entropy_properties}. 

TRAVAILLER SUR PREUVES DE 5 6 7



\subsection{New intuition for supermodularity from entropic volumes}

If one could rigorously prove that an entropy-based set function would give a true measure on the $\sigma$-algebra of majorization cones, then one would immediately have a new proof of supermodularity of Shannon entropy. This is because any measure $\mu$ is monotone (similiar to resource monotonicity), in the sense that for any 2 sets $A, B$, if $A \subseteq B$, then $\mu(A) \leq \mu(B)$. Consider now figure FAIRE FIGURE. It is easy to see that we have

\begin{equation}
    \mathcal{T}_+(p) \cup \mathcal{T}_+(q) \subseteq \mathcal{T}_+(p \wedge q) \implies \mu\left(\mathcal{T}_+(p) \cup \mathcal{T}_+(q)\right) \leq \mu\left(\mathcal{T}_+(p \wedge q)\right).
\end{equation}

\noindent However, by the inclusion-exclusion principle, we have 

\begin{equation}
    \mu\left(\mathcal{T}_+(p) \cup \mathcal{T}_+(q)\right) = \mu\left(\mathcal{T}_+(p)\right) + \mu\left(\mathcal{T}_+(q)\right) - \mu\left(\underbrace{\mathcal{T}_+(p) \cap \mathcal{T}_+(q)}_{= \mathcal{T}_+(p \vee q)}\right),
\end{equation}

\noindent where all the sets are valid majorization cones. If $\mu$ is such that it sends every cone on the entropy of its tip, we would get

\begin{align}
    \mu\left(\mathcal{T}_+(p)\right) + \mu\left(\mathcal{T}_+(q)\right) - \mu\left(\mathcal{T}_+(p \vee q)\right) &\leq \mu\left(\mathcal{T}_+(p \wedge q)\right)\\
    \implies H(p) + H(q) - H(p \vee q) &\leq H(p \wedge q),
\end{align}

\noindent which is precisely the supermodularity property. While we have not shown rigorously that such a set function exists, we believe that the properties of $E^+$ are a good sign that such a measure is possible.



\section{LOCC state selection strategies} \label{sec:strategies}

\subsection{Definition}

Let us assume Alice and Bob are in possession of a bank of pre-shared entangled states $\{q_1, \dots, q_k\} = Q$, and let us assume they are required to use their bank of states for a LOCC protocol, in which they decide through a CC channel on successive target states that they need to construct, e.g. for some form of distributed quantum computing. We also assume that they have at least two different entangled states in their possession, i.e. they could have received some of their entangled states from one provider, and the rest from another provider with a different preparation standard. The problem we are trying to solve is the following. Suppose they decide that they need to construct a target with Schmidt vector $t$, and after looking in their bank, they realize that two states in their bank can reach $t$ through LOCC, e.g. $q_1 \prec t$ and $q_2 \prec t$. Should they use $q_1$ or $q_2$ to construct the target $t$ ?

As far as we know, while research has gone into finding and constructing a state which is the Optimal Common Resource (OCR) of a given set of possible targets $\{t_1, \dots, t_n\}$ in the sense that it can reach all of the possible targets (the OCR of the set is $\wedge_{i=1}^n t_i$), not much research has gone into how to choose between states of a bank that does not exclusively contain OCRs \cite{bosyk_optimal_2019, deside_probabilistic_2024}. If the bank contains states less entangled than the OCR, but that can also reach a given target $t$, then it would be a waste to use up an OCR when a less entangled state could do the job. If several non-OCR states can reach $t$, how do Alice and Bob choose which to use ? We propose the following definition.

\begin{definition}[LOCC state selection strategy]
    A LOCC state selection strategy is any decision algorithm that chooses which state from a pre-shared entanglement bank $Q$ to use to construct a target entangled state $t$ through LOCC, e.g. by minimising some loss function.
\end{definition}

The goal of a good LOCC state selection strategy is then to run out of states capable of reaching successive targets (on which we might have no knowledge over) as slow as possible \textit{on average}. We will assume that after transformation, the LOCC state is consumed (for a quantum computing task the state needs to be measured at the end of the quantum circuit to get a result)\footnote{In practice, the measurement might not need to fully determine the state. For instance, a valid measurement on a ququart could be to measure whether it is in the subspace generated by $\{\ket{0}, \ket{1}\}$ or in the subspace generated by $\{\ket{2}, \ket{3}\}$. Such a measurement would not break a superposition (and thus reduce entanglement of a joint state) if the ququart is in a $\alpha\ket{0} + \beta\ket{1}$ superposition. We will not treat such cases.}.



\subsection{Entropic strategy}

The simplest form of LOCC state selection strategy could be to use the least entangled state that can reach $t$. This simple idea yields the following algorithm, which we use as an example.

\begin{definition}[Entropic strategy]
    Let $Q$ be a bank of pre-shared entangled states, and let $t$ be a target state. The following algorithm decides which state $q \in Q$ to use to construct $t$.
    \begin{enumerate}
        \item Initialize the set $Q' = \emptyset$. For each $q_i \in Q$, if $q_i \succ t$, add $q_i$ to $Q'$, which contains all the states that can reach the target.
        \item For each $q_i \in Q'$, compute $a_i = H(q_i)$.
        \item Finally, construct $t$ using the state $q_i \in Q'$ with the lowest value of $a_i$.
    \end{enumerate}
\end{definition}

This simple strategy essentially uses up the least resourceful state each time. While simple, we believe that more sophisticated strategies might yield better results on average.



\subsection{Volumic strategies}

Let us now turn to the main application of $E^+$. A state $q_i$ with a high unique entropy relative to the rest of the bank $Q \backslash q_i$ is inherently valuable because it can reach many states that the rest of the bank can't reach. If we have no knowledge over the future targets to construct, we would want to avoid using up a state with a high $E^+$ as long as possible, because if we use it up instead of a state that has a low unique entropy, we have a higher risk that at a later step in the protocol a target would fall in the region
$\mathcal{T}_+(q_i) \backslash \left(\cup_{q_j \in Q \backslash q_i} \mathcal{T}_+(q_j)\right)$, which would now be unreachable. However, directly computing the unique entropy of all the states that can reach $t$ and deciding based on that alone is not sufficient. Consider figure FAIRE FIGURE AVEC LA SITUATION DONT ON A PARLE AVEC SERGE. 

Clearly, it would be a waste to use up $q_4$ or $q_5$, because they are majorized by $q_1$ and $q_2$ (respectively), and can therefore reach all of the states that $q_1$ and $q_2$ can reach, respectively. However, $E^+(q_1 \parallel Q \backslash q_1) = E^+(q_2 \parallel Q \backslash q_2) = 0$ because their cones are contained in the cones of $q_4$ and $q_5$, and so without additional restriction on the bank a unique entropy criterion is not sufficient. The approach also fails if we have several copies of the same state, because then all the copies have zero unique entropy. We propose the following LOCC state selection strategy which solves both of these problems.

\begin{definition}[Unique entropy strategy] \label{strat:unique_entropy}
    Let $Q$ be a bank of pre-shared entangled states\footnote{A bank may contain several copies of the same state.}, and let $t$ be a target state. The following algorithm decides which state $q \in Q$ to use to construct $t$.
    \begin{enumerate}
        \item Initialize the set $Q' = \emptyset$. For each $q_i \in Q$, if $q_i \succ t$, add $q_i$ to $Q'$, which contains all the states that can reach the target.
        \item Initialize the set $Q'' = \emptyset$. For each $q_i \in Q'$, if $\nexists q_j \in Q' \: | \: q_i \prec q_j, q_i \neq q_j$ and if $q_i \notin Q''$\footnote{This ensures that if there are several copies of the same state, only one of them enters $Q''$, ensuring that the $E^+$ calculation at the next step does not yield 0 due to the copies.}, add $q_i$ to $Q''$, which contains only the least resourceful states that can reach the target.
        \item For each $q_i \in Q''$, compute $a_i = E^+(q_i \parallel Q'' \backslash q_i)$, and initialize $b_i = 1$; for each $q_j \in Q' \backslash q_i$, if $q_j \prec q_i$, increment $b_i$ by 1. \label{step:volume}
        \item Finally, construct $t$ using the state $q_i \in Q''$ with the lowest value of the ratio $c_i = \frac{a_i}{b_i}$.
    \end{enumerate}
\end{definition}

\noindent It is interesting to note at this stage that this decision algorithm is computationaly more demanding than the entropic strategy.

Essentially, the parameter $b$, which we will call the \textit{redundacy factor}, is used here to take into account the number of times one can reach a region. For example, if one of the high $E^+$ states left in $Q''$ has many copies (or majorized states), then it makes sense to use it to construct the target instead of a lower $E^+$ state with few copies, because using it up makes that region of the bank significantly weaker. Other strategies might choose a different $a$, which we will call the \textit{loss function}. Moreover, the decision criterion is comparing the values of $c$, which we will call the \textit{weighted loss function}, and could also be changed in other strategies\footnote{For example, one could choose $c = \frac{a}{b^2}$, which might yield a different behavior of the strategy in some cases.}. For instance, we propose the following variation of the strategy.

\begin{definition}[Euclidean volume strategy]
    The euclidean volume strategy is the same as the unique entropy strategy (cf. definition \ref{strat:unique_entropy}), except that the loss function is replaced with $a_i = V\left(\mathcal{T}_+(q_i) \backslash \left(\cup_{q_j \in Q'' \backslash q_i} \mathcal{T}_+(q_j)\right)\right)$ in step \ref{step:volume}. The euclidean volume can be computed using the formula
\begin{equation}
    V(\mathcal{T}_+(p)) = , % find gauss area formula
\end{equation}
and using equation (\ref{eq:LOCC_inclusion-exclusion}).
\end{definition}



\subsection{Mixed strategies}

We also define an additional variation, which mixes all three proposed loss functions with a specific ratio which can be adjusted.

\begin{definition}[Mixed strategy]
    Choose $\alpha, \beta \in [0, 1] \: | \: \alpha + \beta \leq 1$. The mixed strategy is the same as the unique entropy strategy (cf. definition \ref{strat:unique_entropy}), except that the loss function is replaced with $a_i = \alpha H(q_i) + \beta E^+(q_i \parallel Q'' \backslash q_i) + (1 - \alpha - \beta) V\left(\mathcal{T}_+(q_i) \backslash \left(\cup_{q_j \in Q'' \backslash q_i} \mathcal{T}_+(q_j)\right)\right)$ in step \ref{step:volume}.
\end{definition}



\subsection{Comparison and statistical sampling}

SIMULER SCENARIO POUR PLUSIEURS VALEURS DE ALPHA ET BETA



\section{Discussion}

