Nielsen and Kempe's main result from their 2001 paper is that if a bipartite quantum state $\rho_{AB}$ is separable, then the vector of eigenvalues of its density matrix (denoted by $\lambda_{AB}$) is majorized by both $\lambda_A$ and $\lambda_B$, the vectors of eigenvalues of the reduced density matrices $\rho_A$ and $\rho_B$. The contrapositive of this statement can be used to detect entanglement: if $\lambda_{AB}$ is not majorized by both $\lambda_A$ and $\lambda_B$, then $\rho_{AB}$ is entangled. Because the Renyi entropies are Schur-concave for all values of $\alpha$, and using the lattice-theoretic notion of meet ($\wedge$) of two probability vectors, this criterion can be rewritten in a weaker form: if $H_\alpha(\lambda_{AB}) < H_\alpha(\lambda_A \wedge \lambda_B)$, then $\rho_{AB}$ is entangled. This is particularly interesting in the case of the collision entropy, which is directly related to the purity of the quantum state by $H_2(\lambda) = S_2(\rho) = -\log(\Tr(\rho^2))$ (with $\gamma = \Tr(\rho^2)$ the purity of the state), which is easier to access than the Shannon entropy of the state as $\rho^2$ is quite easy to compute, whereas computing the vector of eigenvalues can be quite demanding, especially for higher dimensions. It was known that if $\gamma_{AB} > \gamma_A$ or $\gamma_B$, then $\rho_{AB}$ is entangled, but this can now be improved to the following using the lattice: if $\gamma_{AB} > e^{-H_2(\lambda_A \wedge \lambda_B)}$, then $\rho_{AB}$ is entangled. While this is a better lower bound, the usefulness of this statement is not obvious as this is not as strong as the initial majorization criterion.

The question is the following: could we find a density matrix $\rho_{A \wedge B}$ such that $\Tr(\rho^2_{A \wedge B}) = e^{-H_2(\lambda_A \wedge \lambda_B)}$ in a way that is significantly less computationnally intensive than computing the vectors of eigenvalues of the reduced density matrices $\lambda_A$ and $\lambda_B$ and then constructing their meet $\lambda_A \wedge \lambda_B$? Otherwise, one might as well check the majorization criterion directly. If this is not the case, are there other interesting things to be said about $\rho_{A \wedge B}$? Obviously, $\rho_{A \wedge B}$ is not unique (at least up to a unitary matrix), but this is not necessarily an issue. Would any issues arise for constructing $\rho_{A \wedge B}$ if A and B are not of the same dimension?