\begin{abstract}
    The theory of majorization has been used in a variety of fields to compare probability distributions, and has been the key to several theorems in quantum information. In this master thesis, the lattice generated by the majorization relations on the set of n-dimensional probability vectors has been explored, and properties of the lattice have proved to be useful in the study of quantum entanglement. The first result of this thesis is a corollary to Nielsen and Kempe's majorization criterion for separability, which guarantees that a bipartite quantum state is entangled if the vector of eigenvalues of its density matrix has a lower entropy than the meet of the vectors of eigenvalues of its subsystems. This result has been shown to be true not only for the Shannon entropy, but for all Renyi entropies. This is particularly interesting in the case of the collision entropy, which is directly related to the purity of the quantum state, which is easier to access and to compute than the Shannon entropy of the state. The second result of this thesis is the definition of two new complementary measures of comparability between probability vectors defined using Cicalese and Vaccaro's notion of distance between probability vectors on the majorization lattice. This might be useful in a cryptographic setting, where the amount of comparability between the eigenvalues of the two reduced density matrices of a bipartite system might be the source of a new resource theory.
\end{abstract}