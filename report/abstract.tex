%\section*{Abstract}
%\addcontentsline{toc}{chapter}{Abstract}



\begin{abstract}
    The theory of majorization has been used in a variety of fields to compare the certainty of probability distributions. It is fundamentally tied to information theory via a property of Shannon entropy called Schur-concavity, and has been the key to several theorems in quantum information, notably in the field of entanglement transformations using Local Operations and Classical Communications (LOCC), which is a practically relevant protocol for manipulating entangled states shared between two quantum computers. In recent years the majorization lattice, which defines the majorizing supremum (the join) and infimum (the meet) of two distributions, has seen growing interest in the field, enabling the definition of new notions such as an Optimal Common Resource (OCR) in majorization-based Quantum Resource Theories (QRT) such as LOCC. In this master thesis, the majorization lattice of general $d$-dimensional probability distributions is studied, and a novel concatenation-based technique is used to prove a majorization relation underlying a known entropic inequality on the lattice known as subadditivity and generalize it to a broader class of functions. A similar underlying majorization relation, which seems to hold numerically, is also conjectured for supermodularity. The lattice is also used to improve some coupled entropic criteria, which might be relevant in experimental setups for determining whether two mixed states are entangled. The incomparability between two probability distributions is quantified by defining new lattice-based entropic quantities, which are then shown to satisfy many properties expected from an inclusion-exclusion relation, which seems to indicate a new interpretation of Shannon entropy being the volume of a majorization cone. This volumic intuition is finally used to characterize sets of states and to propose LOCC state selection strategies which take into account the entropic volume uniquely LOCC-reachable from a pure state $\ket{\psi}$ and not from any other pure state $\ket{\phi}$ of a pre-shared entanglement bank, which quantifies the redundancy of states in the bank. Numerical simulations indicate that such strategies show a ??? improvement/worsening on average over simpler strategies based on the entanglement resource of each individual state of the bank.
\end{abstract}

\pagenumbering{gobble}