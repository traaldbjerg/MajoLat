\section*{Extended abstract}
%\addcontentsline{toc}{chapter}{Abstract}

\pagenumbering{gobble}

\textbf{Keywords:} Majorization lattice, entanglement, Quantum Resource Theory, incomparability, entropic volume, LOCC

~

Entanglement has long been one of the most puzzling features of quantum mechanics. Some physicists originally believed that entangled states were an artifact of a yet incomplete quantum theory as they deemed some of their properties too absurd for them to truly exist. Notably, it was shown that such states could violate so-called Bell’s inequalities, which must be satisfied by any local hidden-variable model, thereby implying that quantum mechanics – and thus our world – is nonlocal if such states were ever observed experimentally. Such states and the accompanying violation of Bell’s inequalities were finally confirmed experimentally in the last decades of the 20th century, sparking considerable interest in the theory of entanglement.  This led to both deeper theoretical results in the foundations of quantum mechanics and powerful prospects for an emerging quantum technology, which is slowly coming to fruition with, for instance, quantum cryptography, quantum teleportation, or quantum computing.

On the other hand, the mathematical theory of majorization has been used in a variety of fields to compare probability distributions in terms of intrinsic disorder or uncertainty. It provides a (partial) order on probability vectors and is fundamentally tied to information theory via a property of Shannon entropy called Schur-concavity. If a probability vector majorizes another one, it means that is more ordered, implying that its Shannon entropy (as well as every Schur-concave function) must be lower. Majorization theory has found deep applications in quantum information theory, especially in connection with entanglement. It provides a strong separability criterion for detecting whether a state is entangled, which is, in general, a hard problem, and most notably it gives a condition for the interconversion between entangled states via Local Operations and Classical Communications (LOCC), which is a practically relevant class of protocols for manipulating entangled states shared between two quantum computers. The necessary and sufficient condition for a pure bipartite entangled state to be LOCC-convertible into another one can be expressed as a majorization relation implying the so-called Schmidt vectors of the two states. In recent years, the lattice structure induced by the majorization relation – the so-called majorization lattice – has seen growing interest in the field. By drawing on the majorizing supremum (so-called "join") and infimum (so-called "meet") of two probability vectors, the majorization lattice enables a more sophisticated approach to majorization relations with many probability vectors and it is at the heart of majorization-based Quantum Resource Theories (QRT). For example, in a recent work at QuIC, the best conversion protocol to produce the so-called optimal common resource state was proven by using the majorization lattice.

In spite of the close connection between majorization theory and entanglement theory (and some other QRTs), surprisingly little research has been devoted to the majorization lattice since its introduction twenty years ago. The primary objective of this MSc thesis was to address this gap and explore new applications of the majorization lattice to the theory of quantum entanglement. Our starting point was the separability criterion based on a pair of entropy inequalities (comparing the von Neumann entropy of the state to the one of each of its two reduced states). We improved this criterion by leveraging the "meet" of the two reduced states and could prove that the meet-based entropic separability criterion may, in some cases, detect entangled states which remain undetected by the best of the two entropy inequalities. This brought us to the notion of incomparability between two probability vectors (or two bipartite pure entangled states), which arises when a majorization relation linking the two does not exist either way. Incomparability is a key feature in this context because the meet (and join) of two vectors are trivial if the two vectors are comparable. It is therefore tempting to view incomparability as a desirable resource, which is the topic of the core of our MSc thesis.

By taking inspiration from QRTs, we have defined new entropic quantities that measure the amount of incomparability of a probability vector (or bipartite pure entangled state), called probe, with respect to another one, called reference. We introduced a future-incomparability function, which compares the probe with the future majorization cone emerging from the reference (i.e., the set of vectors that majorize the reference). Symmetrically, we considered a past-incomparability function, which compares the probe with the past majorization cone (i.e., the cone of vectors that are majorized by the reference). Importantly, we could prove that these incomparability functions increase or decrease monotonically under a bistochastic degradation of the probe or reference (with one caveat), promoting them to the status of incomparability monotones upon which a proper QRT could hopefully be built. As a relevant quantum application of these notions, we note that the future-incomparability monotone of an entangled state with respect to another one quantifies how much the former state is far from all the states that are LOCC-reachable from the latter state. This hints to the fact that the geometry of the majorization lattice is crucial, though it is still badly understood as of today. 

As an attempt to better understand such a geometry, the last part of our MSc thesis was then centered on a notion of entropic volume in the majorization lattice, which enabled us to extend our resource-theoretical approach of incomparability to a scenario involving multiple probability vectors (or multiple bipartite pure entangled states). The incomparability of a vector with respect to a collection of vectors – what we call a bank – was quantified by using the inclusion-exclusion principle from set theory. The behavior of the proposed generalization seems to indicate that the Shannon entropy of the tip of a majorization cone behaves precisely like the volume of a specific majorization-theoretic set consisting of all LOCC-reachable states from the tip. This intuition was used to quantify the volume of the set of LOCC-reachable states brought by some entangled probe state given a bank of pre-shared entangled states. We call this quantity the uniqueness entropy because it measures what can uniquely be obtained from the probe state in terms of LOCC and is not redundant with what can already be obtained from the bank. While we could not entirely prove that the entropy of the tip of the majorization cone, viewed as a set function, is, indeed, a valid measure, we succeeded in proving at least several of the conditions that follow from measure theory. Furthermore, a convincing argument of it being a measure is that it directly implies the supermodularity of the Shannon entropy on the majorization lattice, which is a proven nontrivial inequality.

Incidentally, as a mathematical side result, our work allowed us to find alternative proofs –  sometimes even generalizations – of fundamental inequalities in the majorization lattice. We could prove the subadditivity of Shannon entropy on the lattice and even generalize it to a larger subclass of Schur-concave functions. With some caveat, we could also prove the supermodularity of Shannon entropy on the majorization lattice and, more interestingly, extend it along the same lines. 

Finally, the applicability of our resource-theoretical approach to incomparability based on the notion of uniqueness entropy was illustrated by constructing some decision algorithms for Resource-State Selection Strategies (RSSS). The goal of the game consists in producing a target entangled state, given a bank of pre-shared entangled states and considering that LOCC protocols are free. The question is to select, among all states of the bank, the one that can produce the target state while consuming the least amount of entanglement, thereby preserving it as much as possible for later use.  By carrying our numerical simulations, we were able to achieve improvements over a simple strategy by using the uniqueness entropy function. While the improvements are not very large, we believe that further research into similar lattice-based quantities might yield even better strategies. More generally, we believe that characterizing entropic distances and volumes in the majorization lattice by exploiting set theory is a fruitful avenue, which will most probably have interesting quantum applications.


%Entanglement has long been one of the most puzzling features of quantum mechanics. Some physicists originally believed that entangled states were impossible as they deemed some of their properties to be too absurd for them to be possible. Notably, it was shown that such states could violate Bell's inequalities that any local hidden-variable model (LHVM) must satisfy, thereby implying that quantum mechanics, and thus our world, is nonlocal if such states were ever confirmed experimentally, which would have considerable philosophical implications. Such states were confirmed experimentally much later, sparking newfound interest in the theory of entanglement which has led to both deeper theoretical results in quantum mechanics and powerful technological prospects which are slowly coming to fruition, such as quantum key distribution, post-quantum cryptography, quantum teleportation, dense conding, \dots

%On the other hand, the theory of majorization has been used in a variety of fields to compare the certainty of probability distributions, by means of the majorization relation which gives a partial order on distributions. It is fundamentally tied to information theory via a property of the Shannon entropy called Schur-concavity, and has been the key to several theorems in quantum information, notably in the field of entanglement transformations using Local Operations and Classical Communications (LOCC), which is a practically relevant protocol for manipulating entangled states shared between two quantum computers. Most notably, the condition for an entangled state to be convertible into another entangled state is stated with a majorization relation between the coefficients of a special decomposition called the Schmidt decomposition. In recent years, the majorization lattice, which defines the majorizing supremum (the join) and infimum (the meet) of two distributions, has seen growing interest in the field. The meet and join, and thus the lattice, enable a more sophisticated study of majorization relations, and they were used to define new notions such as an Optimal Common Resource (OCR) in majorization-based Quantum Resource Theories (QRT). Moreover, an optimal conversion protocol to produce the OCR has very recently been proven using the lattice as well.

%However, given how connected entanglement theory (and some other QRTs) is to majorization, surprisingly little research has been carried out on the majorization lattice since its introduction. The goal of this work was then, given the context, to explore some of the applications of the majorization lattice in the theory of quantum entanglement. The notion that was explored the most was the notion of incomparability between two probability distributions, which is the notion that for any two pair of distributions, a majorization relation linking the two does not necessarily exist. Entropic quantities measuring the amount of incomparability were defined by taking inspiration from QRTs, and some results were also proven that linked these objects to known distributions on the lattice. Interestingly, incomparability also seems to enable some form of diversity in the set of reachable entangled states by LOCC, and seems to be a desirable property for a pair of states. Incomparability also seems to enable improvements for paired entropic criteria by exploiting a condition on the meet, motivating the interest in such a property. The natural question that followed was to generalize these quantities to sets of states, which was done using the inclusion-exclusion principle and successive join operations. The behavior of the proposed generalization seems to naturally imply that the Shannon entropy behaves like the volume of specific majorization-theoretic sets, which happen to precisely be the set of LOCC-reachable states from a given entangled state. This intuition was used to quantify how redundant each of the entangled states in a pre-shared set of entangled states is in terms of LOCC.

%These new notions of incomparability were finally used to define some new decision algorithms for resource-preserving state selection strategies, and improvements over a simple entropic strategy were achieved using the new uniqueness functions. While the improvements are not very large, we believe that further research into similar lattice-based quantities might yield strategies with even better results. 

%In this master thesis, the majorization lattice of general $d$-dimensional probability distributions is studied, and a concatenation-based technique is used to prove a majorization relation underlying a known entropic inequality on the lattice known as subadditivity and generalize it to a broader class of functions. A similar underlying majorization relation, which seems to hold numerically, is also conjectured for supermodularity. The lattice is also used to improve some coupled entropic criteria, which might be relevant in experimental setups for determining whether two mixed states are entangled. The incomparability between two probability distributions is quantified by defining new lattice-based entropic quantities, which are then shown to satisfy many properties expected from an inclusion-exclusion relation, which seems to indicate a new interpretation of Shannon entropy being the volume of a majorization cone. This volumic intuition is finally used to characterize sets of states and to propose LOCC state selection strategies which take into account the entropic volume uniquely LOCC-reachable from a pure state $\ket{\psi}$ and not from any other pure state $\ket{\phi}$ of a pre-shared entanglement bank, which quantifies the redundancy of states in the bank. Numerical simulations indicate that such strategies show a ??? improvement/worsening on average over simpler strategies based on the entanglement resource of each individual state of the bank.



\vspace*{\fill}
\noindent Alexander Stévins\\
Physics Engineering\\
Majorization lattice in the theory of quantum entanglement\\
2024-2025


\pagenumbering{gobble}