\section*{Extended abstract}
%\addcontentsline{toc}{chapter}{Abstract}

\textbf{Keywords:} Majorization lattice, entanglement, Quantum Resource Theory, incomparability, entropic volume, LOCC

~

Entanglement has long been one of the most puzzling features of quantum mechanics. Some physicists originally believed that entangled states were impossible as they deemed some of their properties to be too absurd for them to be possible. Notably, it was shown that such states could violate Bell's inequalities that any local hidden-variable model (LHVM) must satisfy, thereby implying that quantum mechanics, and thus our world, is nonlocal if such states were ever confirmed experimentally, which would have considerable philosophical implications. Such states were confirmed experimentally much later, sparking newfound interest in the theory of entanglement which has led to both deeper theoretical results in quantum mechanics and powerful technological prospects which are slowly coming to fruition, such as quantum key distribution, post-quantum cryptography, quantum teleportation, dense conding, \dots

On the other hand, the theory of majorization has been used in a variety of fields to compare the certainty of probability distributions, by means of the majorization relation which gives a partial order on distributions. It is fundamentally tied to information theory via a property of the Shannon entropy called Schur-concavity, and has been the key to several theorems in quantum information, notably in the field of entanglement transformations using Local Operations and Classical Communications (LOCC), which is a practically relevant protocol for manipulating entangled states shared between two quantum computers. Most notably, the condition for an entangled state to be convertible into another entangled state is stated with a majorization relation between the coefficients of a special decomposition called the Schmidt decomposition. In recent years, the majorization lattice, which defines the majorizing supremum (the join) and infimum (the meet) of two distributions, has seen growing interest in the field. The meet and join, and thus the lattice, enable a more sophisticated study of majorization relations, and they were used to define new notions such as an Optimal Common Resource (OCR) in majorization-based Quantum Resource Theories (QRT). Moreover, an optimal conversion protocol to produce the OCR has very recently been proven using the lattice as well.

However, given how connected entanglement theory (and some other QRTs) is to majorization, surprisingly little research has been carried out on the majorization lattice since its introduction. The goal of this work was then, given the context, to explore some of the applications of the majorization lattice in the theory of quantum entanglement. The notion that was explored the most was the notion of incomparability between two probability distributions, which is the notion that for any two pair of distributions, a majorization relation linking the two does not necessarily exist. Entropic quantities measuring the amount of incomparability were defined by taking inspiration from QRTs, and some results were also proven that linked these objects to known distributions on the lattice. Interestingly, incomparability also seems to enable some form of diversity in the set of reachable entangled states by LOCC, and seems to be a desirable property for a pair of states. Incomparability also seems to enable improvements for paired entropic criteria by exploiting a condition on the meet, motivating the interest in such a property. The natural question that followed was to generalize these quantities to sets of states, which was done using the inclusion-exclusion principle and successive join operations. The behavior of the proposed generalization seems to naturally imply that the Shannon entropy behaves like the volume of specific majorization-theoretic sets, which happen to precisely be the set of LOCC-reachable states from a given entangled state. This intuition was used to quantify how redundant each of the entangled states in a pre-shared set of entangled states is in terms of LOCC.

These new notions of incomparability were finally used to define some new decision algorithms for resource-preserving state selection strategies, and improvements over a simple entropic strategy were achieved using the new uniqueness functions. While the improvements are not very large, we believe that further research into similar lattice-based quantities might yield strategies with even better results. 

%In this master thesis, the majorization lattice of general $d$-dimensional probability distributions is studied, and a concatenation-based technique is used to prove a majorization relation underlying a known entropic inequality on the lattice known as subadditivity and generalize it to a broader class of functions. A similar underlying majorization relation, which seems to hold numerically, is also conjectured for supermodularity. The lattice is also used to improve some coupled entropic criteria, which might be relevant in experimental setups for determining whether two mixed states are entangled. The incomparability between two probability distributions is quantified by defining new lattice-based entropic quantities, which are then shown to satisfy many properties expected from an inclusion-exclusion relation, which seems to indicate a new interpretation of Shannon entropy being the volume of a majorization cone. This volumic intuition is finally used to characterize sets of states and to propose LOCC state selection strategies which take into account the entropic volume uniquely LOCC-reachable from a pure state $\ket{\psi}$ and not from any other pure state $\ket{\phi}$ of a pre-shared entanglement bank, which quantifies the redundancy of states in the bank. Numerical simulations indicate that such strategies show a ??? improvement/worsening on average over simpler strategies based on the entanglement resource of each individual state of the bank.



\vspace*{\fill}
\noindent Alexander Stévins\\
Engineering Physics\\
Majorization lattice in the theory of entanglement\\
2024-2025


\pagenumbering{gobble}