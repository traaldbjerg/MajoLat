\chapter*{Abstract}
\addcontentsline{toc}{chapter}{Abstract}


\textbf{Keywords:} Majorization lattice, Shannon entropy, supermodularity, entanglement, quantum resource theory, incomparability, majorization cone

~

The theory of majorization has been used in a variety of fields to compare probability distributions, and has been the key to several theorems in quantum information. In this master thesis, the lattice generated by the majorization relations on the set of n-dimensional probability vectors has been explored, and properties of the lattice have proved to be useful in the study of quantum entanglement. Several approaches to study the incomparability between two states were explored. New quantities quantifying the incomparability between vectors on the lattice were defined using entropic distances and were shown to be entanglement monotones. Finally, approaches based on the volume of majorization cones were explored, yielding a new method to pick which state to choose out of a bank of states for an entanglement transformation in order to maximize the size of the set of states accessible from the remaining states in the bank.

\vspace*{\fill}
\noindent Alexander Stévins\\
Physics Engineering\\
Majorization lattice in the theory of entanglement\\
2024-2025
